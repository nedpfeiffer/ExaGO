\documentclass[11pt]{report}
%\usepackage{epsf}
\usepackage[left=2.2cm, right=2.2cm, top=2cm]{geometry}

\usepackage[english]{babel} %%% 'french', 'german', 'spanish', 'danish', etc.
\usepackage{amssymb}
\usepackage{amsmath}
\usepackage{breqn}
\usepackage{flexisym}
\usepackage{autobreak}
\usepackage{booktabs}
%\usepackage{txfonts}
\usepackage{verbatim}
\usepackage{longtable}
\usepackage{tabu}
\usepackage{tabularx}
\usepackage{array}

\usepackage{float}
%\usepackage{mathdots}
\usepackage{listings}
\lstset{
  basicstyle = \ttfamily,
  frame = single,
  breaklines = true,
  postbreak=\mbox{\textcolor{red}{$\hookrightarrow$}\space},
}
%\usepackage[classicReIm]{kpfonts}
\usepackage[dvips]{graphicx} %%% use 'pdftex' instead of 'dvips' for PDF output
\usepackage{xcolor}
\usepackage{xspace}
\usepackage{fancyvrb}
\usepackage{courier}
\usepackage[colorlinks]{hyperref}
\usepackage[noabbrev]{cleveref}
\usepackage[toc,page]{appendix}
\usepackage{comment}

\usepackage{tikz}

%\hypersetup{
  %colorlinks=false,
  %linktoc=all
%}

%\restylefloat{table}
%\setlength{\parskip}{1ex}
%\setlength{\parindent}{0ex}
%\newcommand{\half}{{\textstyle{\frac{1}{2}}}}
%\newcommand{\findex}[1]{\index{#1}}
%\newcommand{\sindex}[1]{\index{#1}}
%\newcommand{\F}{\mbox{\boldmath \(F\)}}
%\newcommand{\x}{\mbox{\boldmath \(x\)}}
%\newcommand{\rr}{\mbox{\boldmath \(r\)}}
%\newcommand{\R}{\mathbb R}
%\renewcommand{\Re}{\R}
%\newcommand{\Comment}[1]{}

% Sets and accessors
\newcommand{\Jgen}{J^\text{gen}}
\newcommand{\jinJgen}{j \in \Jgen}
\newcommand{\Jload}{J^\text{ld}}
\newcommand{\jinJload}{j \in \Jload}
\newcommand{\Jbus}{J^\text{bus}}
\newcommand{\iinJbus}{i \in \Jbus}
\newcommand{\Jbr}{J^\text{br}}
\newcommand{\jinJbr}{j \in \Jbr}

% Bounds and parameters
\newcommand{\pminj}{p^\text{gmin}_j}
\newcommand{\pmaxj}{p^\text{gmax}_j}
\newcommand{\qminj}{q^\text{gmin}_j}
\newcommand{\qmaxj}{q^\text{gmax}_j}
\newcommand{\psetj}{p^{\text{gset}}_j}
\newcommand{\thetarefi}{\theta^\text{ref}_i}
\newcommand{\plj}{p^{\text{l}}_j} % Real power load for bus j
\newcommand{\qlj}{q^{\text{l}}_j} % Reactive power load for bus j
\newcommand{\vmini}{v^\text{min}_i}
\newcommand{\vmaxi}{v^\text{max}_i}
\newcommand{\vseti}{v^\text{set}_i}
\newcommand{\srateAj}{s^{\text{rateA}}_j} %% Line j rateA MVA rating

\newcommand{\agj}{a^\text{g}_j} % Generator quadratic cost parameter alpha
\newcommand{\bgj}{b^\text{g}_j} % Generator quadratic cost parameter beta
\newcommand{\cgj}{c^\text{g}_j} % Generator quadratic cost parameter gamma

\newcommand{\sigmalj}{\sigma^\text{l}_j}
\newcommand{\sigmai}{\sigma_i}

\newcommand{\gshi}{g^\text{sh}_i} % shunt conductance
\newcommand{\bshi}{b^\text{sh}_i} % shunt susceptance

% cost
\newcommand{\costgj}{C^\text{g}_j}

% Variables
\newcommand{\pgj}{p^\text{g}_j}
\newcommand{\qgj}{q^\text{g}_j}
\newcommand{\pgjset}{p^{\text{gset}}_j}
\newcommand{\Deltapupj}{\Delta{p}^\text{gu}_j}
\newcommand{\Deltapdownj}{\Delta{p}^\text{gd}_j}
\newcommand{\pmisi}{\Delta{p}_i}
\newcommand{\qmisi}{\Delta{q}_i}
\newcommand{\vreali}{e_i}
\newcommand{\vimagi}{f_i}
\newcommand{\Deltaplj}{\Delta{p}^{\text{l}}_j} % Real power load loss for bus j
\newcommand{\Deltaqlj}{\Delta{q}^{\text{l}}_j} % Reactive power load loss for bus j

\newcommand{\thetai}{\theta_i}
\newcommand{\vi}{v_i}

% Derived and other auxillary terms
\newcommand{\pshi}{p^{\text{sh}}_i} % real power consumed by shunt at bus i
\newcommand{\qshi}{q^{\text{sh}}_i} % reactive power consumed by shunt at bus i
\newcommand{\pbrj}{p^\text{br}_j} % real power flow for line j
\newcommand{\qbrj}{q^\text{br}_j} % real power flow for line j
\newcommand{\pbrjoi}{p^\text{br}_{j_{oi}}} % real power flow on line j, o --> i
\newcommand{\qbrjoi}{q^\text{br}_{j_{oi}}} % reactive power flow on line j, o --> i
\newcommand{\pbrjid}{p^\text{br}_{j_{id}}} % real power flow on line j, i --> d
\newcommand{\qbrjid}{q^\text{br}_{j_{id}}} % reactive power flow on line j, i --> d

\newcommand{\pbrjod}{p^\text{br}_{j_{od}}} % real power flow on line j (o->d) at bus o, o --> d
\newcommand{\qbrjod}{q^\text{br}_{j_{od}}} % reactive power flow on line j (o->d) at bus o, o --> d
\newcommand{\pbrjdo}{p^\text{br}_{j_{do}}} % real power flow on line j (o->d) at bus d, d --> o
\newcommand{\qbrjdo}{q^\text{br}_{j_{do}}} % reactive power flow on line j (o->d) at bus d, d --> o



\newcommand{\option}[1]{\texttt{{#1}}}
\newcommand{\opflowoption}[2]{\texttt{#1 #2}}

% option key
\newcommand{\opflowmodel}{-opflow\_model}
\newcommand{\opflowgensetpoint}{-opflow\_has\_gensetpoint}
\newcommand{\opflowuseagc}{-opflow\_use\_agc}
\newcommand{\opflowincludeloadloss}{-opflow\_include\_loadloss\_variables}
\newcommand{\opflowincludepowerimbalance}{-opflow\_include\_powerimbalance\_variables}
\newcommand{\opflowloadlosspenalty}{-opflow\_loadloss\_penalty}
\newcommand{\opflowpowerimbalancepenalty}{-opflow\_powerimbalance\_penalty}
\newcommand{\opflowobjective}{-opflow\_objective}

% opflow model types
\newcommand{\pbpol}{POWER\_BALANCE\_POLAR}
\newcommand{\pbcar}{POWER\_BALANCE\_CARTESIAN}
\newcommand{\pbpolhiop}{POWER\_BALANCE\_HIOP}
\newcommand{\pbpolrajahiop}{PBPOLRAJAHIOP}

% opflow objective types
\newcommand{\mingencost}{MIN\_GEN\_COST}
\newcommand{\mingensetpointdeviation}{MIN\_GENSETPOINT\_DEVIATION}

% packages
\newcommand{\tao}{\href{https://www.mcs.anl.gov/petsc/documentation/}{TAO \cite{petsc-user-ref}\xspace}}
\newcommand{\ipopt}{\href{https://github.com/coin-or/Ipopt}{Ipopt \cite{ipopt}}}
\newcommand{\hiop}{\href{https://github.com/LLNL/hiop}{HiOp \cite{hiop-manual,hiop1}}}
\newcommand{\raja}{\href{https://github.com/LLNL/RAJA}{RAJA \cite{beckingsale2019umpire}}}
\newcommand{\petsc}{\href{https://www.mcs.anl.gov/petsc/documentation/}{PETSc \cite{petsc-user-ref}}\xspace}
\newcommand{\cmake}{\texttt{CMake}\xspace}
\newcommand{\magma}{\href{https://icl.cs.utk.edu/magma/}{MAGMA}}
\newcommand{\cuda}{\href{https://docs.nvidia.com/cuda/}{CUDA Toolkit }}
\newcommand{\matpower}{\href{https://matpower.org/}{MATPOWER }}




% Does exago always need a 'tm'?
% \newcommand{\exagotm}{ExaGO$\mathrm{{}^{TM}}$\xspace}
% ExaGO applications
\newcommand{\exago}{\texttt{ExaGO}\xspace}
\newcommand{\pflow}{\hyperref[chap:pflow]{PFLOW}\xspace}
\newcommand{\opflow}{\hyperref[chap:opflow]{OPFLOW}\xspace}
\newcommand{\scopflow}{\hyperref[chap:scopflow]{SCOPFLOW}\xspace}
\newcommand{\tcopflow}{\hyperref[chap:tcopflow]{TCOPFLOW} \xspace}
\newcommand{\sopflow}{\hyperref[chap:sopflow]{SOPFLOW} \xspace}

% Miscallenous
\newcommand{\todo}{\textbf{{\textcolor{red}{\Large{**---To be completed---**}}}}}

\makeindex

\begin{document}

\title{\textbf{Exascale Grid Optimization Toolkit (ExaGO$\mathrm{{}^{TM}}$)} \\ \textbf{User Manual} \\ \textbf{{\textcolor{red}{**---Draft---**}}}}
\author{}

\date{\today}
\maketitle

% Table of contents.
\tableofcontents

\newpage
\addcontentsline{toc}{chapter}{License}
\noindent
Copyright © 2020, Batelle Memorial Institute \\
Operator of Pacific Northwest National Laboratory \\
All rights reserved. \\
Exascale Grid Optimization Toolkit (ExaGO), Version \exagoversion \\
OPEN SOURCE LICENSE \\

\noindent
\begin{enumerate}
\item Battelle Memorial Institute (hereinafter Battelle) hereby grants permission to any person or entity lawfully obtaining a copy of this software and associated documentation files (hereinafter “the Software”) to redistribute and use the Software in source and binary forms, with or without modification.  Such person or entity may use, copy, modify, merge, publish, distribute, sublicense, and/or sell copies of the Software, and may permit others to do so, subject to the following conditions:
\begin{itemize}
    \item Redistributions of source code must retain the above copyright notice, this list of conditions and the following disclaimers. 
   \item Redistributions in binary form must reproduce the above copyright notice, this list of conditions and the following disclaimer in the documentation and/or other materials provided with the distribution. 
   \item Other than as used herein, neither the name Battelle Memorial Institute or Battelle may be used in any form whatsoever without the express written consent of Battelle.
\end{itemize}  
\item THIS SOFTWARE IS PROVIDED BY THE COPYRIGHT HOLDERS AND CONTRIBUTORS "AS IS" AND ANY EXPRESS OR IMPLIED WARRANTIES, INCLUDING, BUT NOT LIMITED TO, THE IMPLIED WARRANTIES OF MERCHANTABILITY AND FITNESS FOR A PARTICULAR PURPOSE ARE DISCLAIMED. IN NO EVENT SHALL BATTELLE OR CONTRIBUTORS BE LIABLE FOR ANY DIRECT, INDIRECT, INCIDENTAL, SPECIAL, EXEMPLARY, OR CONSEQUENTIAL DAMAGES (INCLUDING, BUT NOT LIMITED TO, PROCUREMENT OF SUBSTITUTE GOODS OR SERVICES; LOSS OF USE, DATA, OR PROFITS; OR BUSINESS INTERRUPTION) HOWEVER CAUSED AND ON ANY THEORY OF LIABILITY, WHETHER IN CONTRACT, STRICT LIABILITY, OR TORT (INCLUDING NEGLIGENCE OR OTHERWISE) ARISING IN ANY WAY OUT OF THE USE OF THIS SOFTWARE, EVEN IF ADVISED OF THE POSSIBILITY OF SUCH DAMAGE.
\end{enumerate}

This license DOES NOT apply to any third-party libraries used. Those libraries aare covered by their own license.


\noindent
\newpage
********************************************************************************\\
\noindent
DISCLAIMER \\

\noindent
This material was prepared as an account of work sponsored by an agency of the United States Government.  Neither the United States Government nor the United States Department of Energy, nor Battelle, nor any of their employees, nor any jurisdiction or organization that has cooperated in the development of these materials, makes any warranty, express or implied, or assumes any legal liability or responsibility for the accuracy, completeness, or usefulness or any information, apparatus, product, software, or process disclosed, or represents that its use would not infringe privately owned rights.
Reference herein to any specific commercial product, process, or service by trade name, trademark, manufacturer, or otherwise does not necessarily constitute or imply its endorsement, recommendation, or favoring by the United States Government or any agency thereof, or Battelle Memorial Institute. The views and opinions of authors expressed herein do not necessarily state or reflect those of the United States Government or any agency thereof.

\noindent
********************************************************************************\\



% Start of the Users Manual

\chapter{Introduction}\label{chap:intro}
\todo

\chapter{Getting Started}\label{chap:getting_started}

\section{System requirements}

\exago is currently only built on 64b OSX and Linux machines, compiled with GCC $>= 7.3$.
We build \exago on Intel and IBM Power9 architectures.

\section{Prerequisites}

This section assumes that you already have the \exago source code, and that the environment variable \texttt{EXAGODIR} is the directory of the \exago source code.
\exago may be acquired via \href{https://gitlab.pnnl.gov/exasgd/frameworks/exago}{the PNNL git repository linked here}, like so:

\begin{lstlisting}[language=bash]
> git clone https://gitlab.pnnl.gov/exasgd/frameworks/exago.git exago
> export EXAGODIR=$PWD/exago
\end{lstlisting}

Paths to installations of third party software in examples are abbreviated with placeholder paths.
For example, \texttt{/path/to/cuda} is a placeholder for a path to a valid \texttt{CUDA Toolkit} installation.

\section{Dependencies}

\exago has dependencies in table \ref{tab:deps}.

\begin{table}[h]
  \caption{\label{tab:deps}Dependency Table}
  \begin{tabular}{|l|c|c|l|}
    \hline
    \textbf{Dependency} & \textbf{Version Constraints} & \textbf{Mandatory} & \textbf{Notes} \\
    \hline
    \petsc & $>= 3.13.0$ & \checkmark & Only needed for the setup stage \\ \hline
    \cmake & $>= 3.10$ & \checkmark & Only a build dependency \\ \hline
    MPI & $>= 3.1.3$ & & Only tested with \texttt{openmpi} and \texttt{spectrummpi} \\ \hline
    \ipopt & $>= 3.12$ & & \\ \hline
    \hiop & $>= 0.3.0$ & & Prefer dynamically linked \\ \hline
    \raja & $>= 0.11.0$ & & \\ \hline
    \href{https://github.com/LLNL/umpire}{Umpire \cite{umpire}} & $>= 2.1.0$ & & Only when RAJA is enabled \\ \hline
    \magma & $>= 2.5.2$ & & Only when GPU acceleration is enabled \\ \hline
    \cuda & $>= 10.2.89$ & & Only when GPU acceleration is enabled \\
    \hline
  \end{tabular}
  
\end{table}

These may all be toggled via \cmake which will be discussed in the section \hyperref[sec:building_and_installation]{Building and installation}.

\subsection{Notes on environment modules}

Many of the dependencies are available via environment modules on institutional clusters.
To get additional information on your institution's clusters, please ask your institution's system administrators.
Some end-to-end examples in this document will use system-specific modules and are not expected to expected to run on other clusters.

For example, the modules needed to build and run \exago on Newell, an IBM Power9 PNNL cluster, are as follows:

\begin{lstlisting}[language=bash]
> module load gcc/7.4.0
> module load openmpi/3.1.5
> module load cuda/10.2
> module load magma/2.5.2_cuda10.2
> module load metis/5.1.0
> module load cmake/3.16.4
\end{lstlisting}

\subsection{Additional Notes on GPU Accelerators}

As of January 2021, \texttt{CUDA} is the only GPU accelerator platform \exago fully supports.
We have preliminary support for \texttt{HIP}, however this is not in any of our main development branches yet.
Full \texttt{HIP} support should arrive in early 2021.

\subsection{Additional Notes on Umpire}

\texttt{Umpire} is an implicit dependency of \texttt{RAJA}.
If a user enables \texttt{RAJA}, they must also provide a valid installation of \texttt{Umpire}.
Additionally, if a user would like to run \exago with \texttt{RAJA} and without \texttt{CUDA}, they must provide a CPU-only build of \texttt{Umpire} since an \texttt{Umpire} build with \texttt{CUDA} enabled will link against \texttt{CUDA}.

\section{Building and installation}
\label{sec:building_and_installation}

\subsection{Default Build}

\exago may be built with a standard \cmake workflow:

\begin{lstlisting}[language=bash,caption={Example \cmake workflow}]
> cd $EXAGODIR
> export BUILDDIR=$PWD/build INSTALLDIR=$PWD/install
> mkdir $BUILDDIR $INSTALLDIR
> cd $BUILDDIR
> cmake .. -DCMAKE_INSTALL_PREFIX=$INSTALLDIR
> make install
\end{lstlisting}

Following sections will assume the user is following the basic workflow outlined above.

\textbf{Note:} For changes to the \cmake configuration to take effect, the code will have to be rebuilt.

\subsection{Additional Options}

To enable additional options, \cmake variables may be defined via \cmake command line arguments, \texttt{ccmake}, or \texttt{cmake-gui}.
\cmake options specific to \exago have an \texttt{EXAGO\_} prefix.
For example, the following shell commands will build \exago with \texttt{MPI}:

\begin{lstlisting}[language=bash]
> cmake .. -DCMAKE_INSTALL_PREFIX=$INSTALLDIR -DEXAGO_ENABLE_MPI=ON
\end{lstlisting}

\exago's \cmake configuration will search the usual system locations for an \texttt{MPI} installation.

For dependencies not installed to a system-wide location, users may also directly specify the location of a dependency.
For example, this will build \exago with \texttt{IPOPT} enabled and installed to a user directory:

\begin{lstlisting}[language=bash]
> cmake .. \
    -DCMAKE_INSTALL_PREFIX=$INSTALLDIR \
    -DEXAGO_ENABLE_IPOPT=ON \
    -DIPOPT_DIR=/path/to/ipopt
\end{lstlisting}

Notice that the \cmake variable \texttt{IPOPT\_DIR} does not have an \texttt{EXAGO\_} prefix.
This is because the variables specifying locations often belong to external \cmake modules.
\cmake variables indicating installation directories do not have an \texttt{EXAGO\_} prefix.

Some \cmake options effect others.
This is especially common when the user enables \exago's GPU options.
For example, if the user enables \texttt{EXAGO\_ENABLE\_GPU} and \texttt{EXAGO\_ENABLE\_RAJA}, the user must provide a GPU-enabled \texttt{RAJA} installation.
\texttt{Umpire} is also an implicit dependency of \texttt{RAJA}, so if the user enables \texttt{EXAGO\_ENABLE\_GPU} they must \textbf{also} provide a GPU-enabled \texttt{Umpire} installation.

Below is a complete shell session on PNNL's cluster Newell in which a more complicated \exago configuration is built, where each dependency installation is explicitly passed to \cmake.
Environment modules specific to Newell are provided to make the example thorough, even though they are not likely to work on another machine.

\begin{lstlisting}[language=bash,caption={\exago build with all options enabled}]
> module load gcc/7.4.0
> module load cmake/3.16.4
> module load openmpi/3.1.5
> module load magma/2.5.2_cuda10.2
> module load metis/5.1.0
> module load cuda/10.2
> git clone https://gitlab.pnnl.gov/exasgd/frameworks/exago.git exago
> export EXAGODIR=$PWD/exago
> cd $EXAGODIR
> export BUILDDIR=$PWD/build INSTALLDIR=$PWD/install
> mkdir $BUILDDIR $INSTALLDIR
> cd $BUILDDIR
> cmake .. \
  -DCMAKE_INSTALL_PREFIX=$INSTALLDIR \
  -DCMAKE_BUILD_TYPE=Debug \
  -DEXAGO_ENABLE_GPU=ON \
  -DEXAGO_ENABLE_HIOP=ON \
  -DEXAGO_ENABLE_IPOPT=ON \
  -DEXAGO_ENABLE_MPI=ON \
  -DEXAGO_ENABLE_PETSC=ON \
  -DEXAGO_RUN_TESTS=ON \
  -DEXAGO_ENABLE_RAJA=ON \
  -DEXAGO_ENABLE_IPOPT=ON \
  -DIPOPT_DIR=/path/to/ipopt \
  -DRAJA_DIR=/path/to/raja \
  -Dumpire_DIR=/path/to/umpire \
  -DHIOP_DIR=/path/to/hiop \
  -DMAGMA_DIR=/path/to/magma \
  -DPETSC_DIR=/path/to/petsc
> make -j 8 install
> # For the following commands, a job scheduler command may be needed.
> # Run test suite
> make test
> # Run an ExaGO application:
> $INSTALLDIR/bin/opflow
\end{lstlisting}

\section{Usage}

Each \exago application has the following format for execution
\begin{lstlisting}[language=bash]
  ./app <app_options>
\end{lstlisting}
Here, \lstinline{app_options} are the command line options for the application. Each application has many options through which the input files and the control options can be set for the application. All application options have the form \lstinline{-app_option_name} followed by the \lstinline{app_option_value}. 
For instance,
\begin{lstlisting}[language=bash]
  ./opflow -netfile case9mod.m -opflow_model POWER_BALANCE_POLAR \
  -opflow_solver IPOPT
\end{lstlisting}
will execute \opflow application using \lstinline{case9mod.m} input file with the model \lstinline{POWER_BALANCE_POLAR} and \ipopt solver.

Options can also be passed to each application through \lstinline{-options_file}, or through a combination of the command line and options file. The configuration specified last on the command line overrides any previous options. For example, if \lstinline{-options_file opflowoptions} specified \lstinline{-netfile case9mod.m} within it's settings:
\begin{lstlisting}[language=bash]
./opflow -netfile case118.m -options_file opflowoptions # Uses case9mod.m
./opflow -options_file opflowoptoins -netfile case118.m # Uses case118.m
\end{lstlisting}
If no options file is specified through the command line, ExaGO applications will attempt to locate the default options file for a given application in \lstinline{.,./options,<install_dir>/options}.
Each solver that can be used within \exago also have their own unique options. Some of these options are specified internally within \exago, but any other options that are needed should either be passed through the command line, or additionally specified in the options file.


\chapter{Optimal power flow (OPFLOW)}\label{chap:opflow}

OPFLOW solves the full AC optimal power flow problem and provides various flexible features that can be toggled via run-time options. It has interfaces to different optimization solvers that can be executed on CPUs or on GPUs.

\section{Formulation}
Optimal power flow is a general nonlinear programming problem with the following form
\begin{align}
\text{min. }& f(x) \\
&\text{s.t.} \nonumber \\
& g(x) = 0 \\
& h(x) \le 0 \\
& x^{\text{min}} \le x \le x^{\text{max}}
\end{align}
Here, $x$ are the decision variables with lower and upper bounds $x^{\text{min}}$ and $x^{\text{max}}$, respectively, $f(x)$ is the objective function, $g(x)$ and $h(x)$ are the equality and inequality constraints, respectively. In the following sections we describe what constitutes these different terms as used by OPFLOW.

\subsection{Variables and bounds} \label{subsec:opflow_var}

The different variables used in \opflow formulation are described in Table \ref{tab:opflow_vars}.

\begin{table}[!htbp]
\caption{Optimal power flow (OPFLOW) variables}
\small
  \begin{tabular}{|p{0.1\textwidth}|p{0.2\textwidth}|p{0.2\textwidth}|p{0.5\textwidth}|}
   \hline
    \textbf{Symbol} & \textbf{Variable} & \textbf{Bounds} & \textbf{Notes} \\
    \hline
    $\pgj$ & Generator real power dispatch & $\pminj \le \pgj \le \pmaxj$ & ~\\
    \hline
    $\qgj$ & Generator reactive power dispatch & $\qminj \le \qgj \le \qmaxj$ & ~ \\
    \hline
    $\Deltapj$ & Generator real power deviation & $-p^{\text{r}}_j \le \Deltapj \le p^{\text{r}}_j$ & \begin{itemize}[noitemsep,topsep=0pt,leftmargin=*]  \item Only used when \option{\opflowgensetpoint} or \option{\opflowuseagc} option is active \item $\Deltapj$ is the deviation from real power generation setpoint $\psetj$.\end{itemize} \\
    \hline
    $\pgjset$ & Generator real power set-point & $\pminj \le \pgjset \le \pmaxj$ & \begin{itemize}[noitemsep,topsep=0pt,leftmargin=*] \item Only used when \option{\opflowgensetpoint} or \option{\opflowuseagc} option is active. \end{itemize} \\
    \hline
    $\Delta{P}$ & System power excess/deficit & Unbounded & Only used when \option{\opflowuseagc} is active \\
    \hline
    $\thetai$ & Bus voltage angle & -$\pi \le \thetai \le \pi$ & 
    \begin{itemize}[noitemsep,topsep=0pt,leftmargin=*] 
    	\item Used with power balance polar model (\option{\opflowmodel~\pbpol}) 
	\item $\thetai$ is unbounded, except reference bus angle $\thetarefi$ which is fixed to 0 
    \end{itemize} \\
    \hline
    $\vi$ & Bus voltage magnitude & $\vmini \le \vi \le \vmaxi$ & \begin{itemize}[noitemsep,topsep=0pt,leftmargin=*] \item Used with power balance polar model (\option{\opflowmodel~\pbpol})\item $\vmini = \vmaxi = \vseti$ if fixed generator set point voltage option is active (\option{\opflowgensetpoint}) \end{itemize} \\
    \hline
  %  Bus voltage real part & $\vreali$ & $-\vmaxi \le \vreali \le \vmaxi$ & Used with power balance cartesian model (\option{\opflowmodel~ \pbcar})\\
  %  \hline
  %  Bus voltage imaginary part & $\vimagi$ & $-\vmaxi \le \vimagi \le \vmaxi$ & Used with power balance cartesian model (\option{\opflowmodel~\pbcar})\\
  %  \hline
    $\pmisplusi,\pmisminusi$ & Bus real power mismatch variables & $0 \le \pmisplusi,\pmisminusi \le \infty$ & Used when power mismatch variable option is active (\option{\opflowincludepowerimbalance 1}) \\
    \hline
    $\qmisplusi,\qmisminusi$ & Bus reactive power mismatch variables & $0 \le \qmisplusi,\qmisminusi \le \infty$ & Used when power mismatch variable option is active (\option{\opflowincludepowerimbalance 1}) \\
    \hline
    $\Deltaplj$ & Real power load loss & $0 \le \Deltaplj \le \plj $ & Used when load loss variable option is active (\option{\opflowincludeloadloss 1}) \\
    \hline
    $\Deltaqlj$ & Reactive power load loss & $0 \le \Deltaqlj \le \qlj$ & Used when load loss variable option is active (\option{\opflowincludeloadloss 1}) \\
    \hline
  %  Bus reactive power mismatch & $q^{mis}_i$ & Unbounded & Used when power mismatch variable option is active (\text{-opflow_include_powerimbalance_variables 1})
  %  \hline
  \end{tabular}
  \label{tab:opflow_vars}
\end{table}
Power imbalance variables are non-physical (slack) variables that measure the violation of power balance at buses. Having these variables (may) help in making the optimization problem easier to solve since they always ensure feasibility of the bus power balance constraints.

\subsection{Objective Function}\label{sec:opflow_obj}

The objective function for OPFLOW is given in (\ref{eq:opflow_obj})
\begin{equation}
\text{min.} ~ C_{gen}(p^{\text{g}}) + C_{dev}(\Delta p^{\text{g}}) +
C_{loss}(\Delta p^{\text{l}},\Delta q^{\text{l}}) + C_{imb}(\Delta p^{+},\Delta p^{-},\Delta q^{+},\Delta q^{-})
\label{eq:opflow_obj}
\end{equation}
 
\subsubsection{Total generation cost $C_{gen}(p^{\text{g}})$}
Needs option \opflowoption{\opflowobjective}{\mingencost}
\begin{equation}
C_{gen}(p^{\text{g}}) = \sum_{\jinJgen} \costgj(\pgj)
\label{eq:opf_obj_mingencost}
\end{equation}
Here, $\costgj$ is a quadratic function of the form $\costgj = \agj{\pgj}^2 + \bgj\pgj + \cgj$.

\subsubsection{Total generation setpoint deviation $C(\Delta p^{\text{g}})$}
Needs option \opflowoption{\opflowobjective}{\mingensetpointdeviation}
\begin{equation}
C_{dev}(\Delta p^{\text{g}}) = \sum_{\jinJgen} ({\Deltapj}^2)
\label{eq:opf_obj_mingensetpointdev}
\end{equation}
This feature is only supported with IPOPT solver.

\subsubsection{Load loss $C(\Delta p^{\text{l}},\Delta q^{\text{l}})$}
This term gets added to the objective when  \option{\opflowincludeloadloss} option is active. 
\begin{equation}
C_{loss}(\Delta p^{\text{l}},\Delta q^{\text{l}}) =  \sum_{\jinJload} \sigmalj({\Deltaplj} + {\Deltaqlj})
\label{eq:opf_obj_minloadloss}
\end{equation}
The load loss penalty $\sigmalj$ can be set via the option
\option{-opflow\_loadloss\_penalty}. The default is \$1000/MW for all loads.

\subsubsection{Power imbalance $C_{imb}(\Delta p^{+},\Delta p^{-},\Delta q^{+},\Delta q^{-})$}
This term gets added to the objective when  \option{-opflow\_include_powerimbalance_variables} option is active. 
\begin{equation}
C_{imb}(\Delta p^{+},\Delta p^{-},\Delta q^{+},\Delta q^{-}) =  \sum_{i \in J^{bus}} \sigmai(\Delta p^{+} + \Delta p^{-} + \Delta q^{+} + \Delta q^{-})
\label{eq:opf_obj_minpowerimbalance}
\end{equation}
The power imbalance cost $\sigmai$ can be set via the option
\option{-opflow\_powerimbalance\_penalty}. The default is \$10,000/MW$^2$ for all buses. Though the power imbalance variables $\Delta p^{+},\Delta p^{-},\Delta q^{+},\Delta q^{-}$ are slack or non-physical, they can help in solving infeasible cases having no power flow solution, and thus provide a measure of the infeasibility. 

\subsection{Equality constraints}\label{sec:opflow_eq}

\subsubsection{Nodal power balance}
The nodal power balance equations for each bus $i$ are given by
\begin{align}
\sum_{\substack{\jinJgen \\ A^{\text{g}}_{ij} \neq 0}} \pgj &=   \pshi + \pmisplusi - \pmisminusi + \sum_{\substack{\jinJload \\ A^{\text{l}}_{ij} \neq 0}} (\plj - \Deltaplj) + \sum_{\substack{\jinJbr \\ A^{\text{br}}_{oi} \neq 0}} \pbrjod + \sum_{\substack{\jinJbr \\ A^{\text{br}}_{id} \neq 0}} \pbrjdo \\
\sum_{\substack{\jinJgen \\ A^{\text{g}}_{ij} \neq 0}} \qgj &=  \qshi + \qmisplusi - \qmisminusi + \sum_{\substack{\jinJload \\ A^{\text{l}}_{ij} \neq 0}} (\qlj - \Deltaqlj) +
\sum_{\substack{\jinJbr \\ A^{\text{br}}_{oi} \neq 0}} \qbrjod + \sum_{\substack{\jinJbr \\ A^{\text{br}}_{id} \neq 0}} \qbrjdo \\
\end{align}
where, the real and reactive power shunt consumption is given by (\ref{eq:opflow_sh_p}) and (\ref{eq:opflow_sh_q})


The real and reactive power flow $\pbrjod$, $\qbrjod$ for line $j$ from the origin bus $o$ to destination bus $d$ is given by (\ref{eq:opflow_br_podflow}) -- 
 (\ref{eq:opflow_br_qodflow})
and from destination bus $d$ to origin bus $o$ is given by (\ref{eq:opflow_br_pdoflow}) -- 
 (\ref{eq:opflow_br_qdoflow})

 \subsubsection{Shunt power}
\begin{align}
&\pshi = \gshi{\vi}^2 \label{eq:opflow_sh_p} \\
&\qshi = -\bshi{\vi}^2 \label{eq:opflow_sh_q}
\end{align}

\subsubsection{Generator real power output}

When using \option{\opflowgensetpoint}, two extra variables $\pgjset$ and $\Deltapj$ are added for each generator. The generator real power output $\pgj$ is related to the power deviation $\Deltapj$ by the following relations
\begin{align}
  \pgjset + \Deltapj - \pgj = 0 \\
  \pgjset - p^{\text{g*}}_j = 0
\end{align}
The second equation sets the generator set-point $\pgjset$ to a fixed value $p^{\text{g*}}_j$. Here, $p^{\text{g*}}_j$ is the set-point for the generator real power output, which can be thought of as an operator set or contractual agreement set-point.

\subsection{Inequality constraints}

\subsubsection{MVA flow on branches}
MVA flow limits at origin and destination buses for each line.
\begin{align}
  {\pbrjod}^2 + {\qbrjod}^2 \le {\srateAj}^2,~~\jinJbr\\
  {\pbrjdo}^2 + {\qbrjdo}^2 \le {\srateAj}^2,~~\jinJbr
\end{align}
To reduce the number of inequality constraints, only lines that are in service and having MVA A rating $\srateAj$ less than 10000 MVA are considered.

\subsubsection{Branch flows}
In polar coordinates, the real and reactive power flow $\pbrjod$, $\qbrjod$  from bus $o$ to bus $d$ on line $j$ is given by (\ref{eq:opflow_br_podflow}) -- 
 (\ref{eq:opflow_br_qodflow})


\begin{align}
\pbrjod &= g_{oo}{v^2_o} + v_ov_d(g_{od}\cos(\theta_o - \theta_d) + b_{od}\sin(\theta_o - \theta_d)) \label{eq:opflow_br_podflow}\\
\qbrjod &= -b_{oo}{v^2_o} + v_ov_d(-b_{od}\cos(\theta_o - \theta_d) + g_{od}\sin(\theta_o - \theta_d)) \label{eq:opflow_br_qodflow}
\end{align}

\noindent
and from bus $d$ to bus $o$ is given by (\ref{eq:opflow_br_pdoflow}) -- (\ref{eq:opflow_br_qdoflow})

\begin{align}
\pbrjdo &= g_{dd}{v^2_d} + v_dv_o(g_{do}\cos(\theta_d - \theta_o) + b_{do}\sin(\theta_d - \theta_o))  \label{eq:opflow_br_pdoflow} \\
\qbrjdo &= -b_{dd}{v^2_d} + v_dv_o(-b_{do}\cos(\theta_d - \theta_o) + g_{do}\sin(\theta_d - \theta_o)) \label{eq:opflow_br_qdoflow}
\end{align}

\subsubsection{Automatic generation control (AGC)}
With \option{\opflowuseagc}, two additional constraints are added for each participating generator to enforce the proportional generator redispatch participation as done in automatic generation control (AGC). These two equations are 
\begin{equation}
\begin{aligned}
  \left(\alpha^{\text{g}}_j\Delta{P} - \Deltapj\right)\left(\pgj - \pmaxj\right) \ge 0 \\
  \left(\Deltapj - \alpha^{\text{g}}_j\Delta{P}\right)\left(\pminj - \pgj\right) \ge 0
\end{aligned}
\label{eq:opflow_agc}
\end{equation}
Eq. \ref{eq:opflow_agc} forces the generator set-point deviation to be equal to the generation participation when the generator has head-room available $\pminj \le \pgj \le \pmaxj$. Here, $\alpha^{\text{g}}_j$ is the generator participation factor which is the proportion of the power deficit/excess $\Delta{P}$ that the generator provides.

\subsubsection{Generator bus voltage control}
When the option \opflowoption{\opflowgenbusvoltage~\fixedwithinqbounds} is used, the generator bus voltage is fixed when the total reactive power generation available at the bus is within bounds. When it reaches its bounds, the voltage varies with the generator reactive power fixed at its bound. To implement this behavior, two inequality constraints are added for each generator bus
\begin{equation}
\begin{aligned}
(v^{\text{set}}_i - \vi)(q_i - q^{\text{max}_i}) \ge 0 \\
(\vi - v^{\text{set}}_i)(q^{\text{min}_i} - q_i) \ge 0
\end{aligned}
\label{eq:opflow_genbusvoltage}
\end{equation}
Here, $q_i$, $q^{\text{max}_i}$, and $q^{\text{min}_i}$ are the generated, maximum, and minimum reactive power at the bus, respectively.

\begin{comment}
\subssubection{Voltage magnitude for cartesian coordinates}
When using cartesian coordinates for voltages, inequality constraints (\ref{eq:opflow_ineq_vmag}) need to introduced to constraining the voltage magnitude within its bounds
\begin{equation}
  {\vmini}^2 \le {\vreali}^2 + {\vimagi}^2 \le {\vmaxi}^2,~~\iinJbus
\label{eq:opflow_ineq_vmag}
\end{equation}
\end{comment}

\section{Solvers}\label{sec:opflow_solvers}
\opflow can be used with a few different solvers. All the solvers solve the optimization problem via a nonlinear interior-point algorithm.
\begin{enumerate}
  \item \ipopt~is a popular open-source package for solving nonlinear optimization problems. It is the most robust of the solvers implemented for solving \opflow. However, it can be run only on a single process and does not have GPU support. \\ Option:
  \opflowoption{-opflow\_solver}{IPOPT} \opflowoption{-opflow_model}{POWER\_BALANCE\_POLAR}

  \item \hiop~ is a high-performance optimization library that implements an interior-point algorithm for solving nonlinear optimization problems. There are two solvers available from the \hiop library: Mixed sparse-dense formulation {\opflowoption{-opflow\_solver}{HIOP}}, and sparse formulation {\opflowoption{-opflow\_solver}{HIOPSPARSE}}. The library supports execution both on the CPU and the GPU. Options: \\ CPU: {\opflowoption{-opflow\_solver}{HIOP}} {\opflowoption{-opflow\_model}{POWER\_BALANCE\_HIOP}} {\opflowoption{-hiop\_compute\_mode}{CPU}} \\ GPU: \opflowoption{-opflow\_solver}{HIOP} {\opflowoption{-opflow\_model}{PBPOLRAJAHIOP}} {\opflowoption{-hiop\_compute\_mode}{GPU}} 

 % \item The Toolkit for Applied Optimization (\tao) is a toolkit available through \petsc for solving optimization problems. TAO includes a interior-point algorithm that is used by OPFLOW. The TAO algorithm is fully parallel, but is less robust compared to IPOPT. Currently, the TAO solver can only be used with the power balance cartesian model. Option:\\ \opflowoption{-opflow\_solver TAO} \opflowoption{-opflow\_model POWER\_BALANCE\_CARTESIAN}
  %The option \opflowoption{-hiop\_ipopt\_debug} can be used with the CPU option in order to use an adapter that verifies the correctness of the first and second order derivates when compared to \ipopt. This may be useful when debugging experimental changes.

\end{enumerate}

\section{Models}\label{sec:opflow_model}

A `model' in \exago describes the representation of the underlying physics. All \opflow models use the power balance formulation in polar coordinates for the ACOPF equations. The difference between the different models arises from their specific implementation/interface. The different models available for \opflow are listed in Table \ref{tab:opflow_models}. As discussed earlier, not every `model' is compatible with every `solver'. Table \ref{tab:opflow_model_solver_compatibility} lists the solver compatibility for the different models.

\begin{table}[!h]
  \caption{OPFLOW models}
  \small
  \begin{tabular}{|p{0.2\textwidth}|p{0.4\textwidth}|p{0.2\textwidth}|p{0.15\textwidth}|}
    \hline
    \textbf{Model type} & \textbf{OPFLOW option (\opflowoption{\opflowmodel}{})} & \textbf{Compatible solvers} & \textbf{CPU-GPU}\\
    \hline
    Power balance with polar coordinates & \pbpol & IPOPT, HIOPSPARSE & CPU\\
    \hline
%    Power balance with cartesian coordinates & \pbcar & IPOPT, TAO & CPU\\
%    \hline
    Power balance with polar coordinates used with HIOP & {\pbpolhiop} & \hiop & CPU/GPU\\
    \hline
    Power balance with polar coordinates used with HIOP on GPU & {\pbpolrajahiop} & \hiop & GPU\\
    \hline
  \end{tabular}
  \label{tab:opflow_models}
\end{table}

\begin{table}[h!]
  \centering
  \caption{OPFLOW Model-solver compatibility}
  \begin{tabular}{|c|c|c|c|}
    \hline
    Model Name & \ipopt & \hiop & HIOPSPARSE \\ \hline
    POWER\_BALANCE\_POLAR & \checkmark & & \checkmark \\ \hline
%    POWER\_BALANCE\_CARTESIAN & \checkmark &  & \checkmark & \\ \hline
    POWER\_BALANCE\_HIOP & & \checkmark &  \\ \hline
    PBPOLRAJAHIOP & & \checkmark &  \\ \hline
  \end{tabular}
\label{tab:opflow_model_solver_compatibility}
\end{table}

\subsection{Power balance polar}
The power balance polar model ({\opflowmodel~\pbpol}) uses the power balance formulation with polar representation for the network voltages. It runs on CPU only and is compatible with \ipopt and sparse \hiop solvers.

\begin{comment}
\subsection{Power balance cartesian}
\end{comment}

\subsection{Power balance with HiOp on CPU}
This model ({\opflowmodel~\pbpolhiop}) implements the power balance formulation with polar coordinates used with \hiop solver only. The model evaluation is done only on the CPU, but the \hiop solver can be executed either on the CPU (\opflowoption{-hiop\_compute_mode}{CPU}) or GPU (\opflowoption{-hiop\
_compute\_mode}{HYBRID}) by setting the \option{-hiop_compute_mode} option appropriately.

\subsection{Power balance with HiOp on GPU}
The PBPOLRAJAHIOP model ({\opflowmodel~\pbpolrajahiop}) computes all the model and optimization calculations on the GPU. This model uses \raja and Umpire \cite{beckingsale2019umpire} libraries to run \opflow calculations (objective, constraints, etc.) on the GPU. 

\section{Input and Output}
The current \exago version only supports reading network files in \matpower format and can (optionally) write the output back in \matpower data file format.

\section{Usage}
\begin{lstlisting}
  ./opflow -netfile <netfilename>  <opflowoptions>
\end{lstlisting}

\section{Options}
See table \ref{tab:opflow_options}
\begin{table}[!htbp]
  \caption{OPFLOW options}
  \small
  \begin{tabular}{|p{0.4\textwidth}|p{0.2\textwidth}|p{0.2\textwidth}|p{0.2\textwidth}|}
    \hline
    \textbf{Option} & \textbf{Meaning} & \textbf{Values (Default value)} & \textbf{Compatibility}\\ \hline
    -netfile & Network file name & string $<$ 4096 characters (\href{https://gitlab.pnnl.gov/exasgd/frameworks/exago/-/blob/master/datafiles/case9/case9mod.m}{case9mod.m}) & \\ \hline
    -print\_output & Print output to screen & 0 or 1 (0) & All solvers\\ \hline
    -save\_output & Save output to file & 0 or 1 (0) & All solvers \\ \hline
    -opflow\_model & Representation of network balance equations and bus voltages & See Table \ref{sec:opflow_model} (POWER\_BALANCE\_POLAR) & \\ \hline
    -opflow\_solver & Optimization solver & See section \ref{sec:opflow_solvers} & \\ \hline
    -opflow\_initialization & Type of initialization & See Table \ref{tab:opflow_initializations} (MIDPOINT) & All solvers \\ \hline
    -opflow\_has\_gensetpoint & Uses generation set point and activates ramping variables & 0 or 1 (0) & All models\\ \hline
    -opflow\_use\_agc & Uses AGC formulation in OPF & 0 or 1 (0) & POWER\_BALANCE \_POLAR only \\ \hline
    -opflow\_objective & type of objective function & See table \ref{tab:opflow_objtypes} (MIN\_GEN\_COST) & All models\\ \hline
    -opflow\_genbusvoltage & Type of generator bus voltage control & See Table \ref{tab:opflow_genbusvoltage} (VARIABLE\_WITHIN \_BOUNDS) & POWER\_BALANCE \_POLAR only \\ \hline
    -opflow\_ignore\_lineflow\_constraints & Ignore line flow constraints & 0 or 1 (0) & All models \\ \hline
    -opflow\_include\_loadloss\_variables & Include load loss & 0 or 1 (0) & All models\\ \hline
    -opflow\_include\_powerimbalance\_variables & Include power imbalance & 0 or 1 (0) & All models \\ \hline
    -opflow\_loadloss\_penality & \$ penalty for load loss & real (1000) & All models \\ \hline
    -opflow\_powerimbalance\_penalty & \$ penalty for power imbalance & real (10000) & All models \\ \hline
    -opflow\_tolerance & Optimization solver tolerance & real (1e-6) & All solvers \\ \hline 
  \end{tabular}
  \label{tab:opflow_options}
\end{table}

\begin{table}[!htbp]
  \centering
  \caption{OPFLOW initializations}
  \begin{tabular}{|c|c|}
    \hline
    \textbf{Initialization type} & \textbf{Meaning} \\ \hline
    MIDPOINT & Use mid-point of bounds \\ \hline
    FROMFILE & Use values from network file \\ \hline
    ACPF & Run AC power flow for initialization \\ \hline
    FLATSTART & Flat-start \\ \hline
  \end{tabular}
\label{tab:opflow_initializations}
\end{table}

\begin{table}[!htbp]
  \centering
  \caption{OPFLOW generator bus voltage control modes}
  \begin{tabular}{|p{0.35\textwidth}|p{0.3\textwidth}|p{0.35\textwidth}|}
    \hline
    \textbf{Voltage control type} & \textbf{Meaning} & \textbf{Compatibility}\\ \hline
%    FIXED\_AT\_SETPOINT & Fixed at given set-point. Reactive power limits are ignored \\ \hline
    FIXED\_WITHIN\_QBOUNDS & Fixed within reactive power bounds & POWER\_BALANCE\_POLAR only \\ \hline
    VARIABLE\_WITHIN\_BOUNDS & Variable within voltage bounds & All models \\ \hline
  \end{tabular}
\label{tab:opflow_genbusvoltage}
\end{table}

\begin{table}[!htbp]
  \centering
  \caption{OPFLOW objective function types}
  \begin{tabular}{|p{0.35\textwidth}|p{0.35\textwidth}|p{0.3\textwidth}|}
    \hline
    \textbf{Objective function} & \textbf{Meaning} & \textbf{Compatibility}\\ \hline
    MIN\_GEN\_COST & Minimize generation cost & All models \\ \hline
    MIN\_GENSETPOINT\_DEVIATION & Minimize deviation (ramp up-down) from generataor set-point & POWER\_BALANCE\_POLAR model only\\ \hline
    NO\_OBJ & No objective function (only feasibility) & All models \\ \hline
  \end{tabular}
\label{tab:opflow_objtypes}
\end{table}


\section{Examples}

Some \opflow example runs are provided with some sample output. Options values are the default values in table \ref{tab:opflow_options} unless otherwise specified. \opflowoption{-print_output} is only used in the first example to save space. Sample output is generated by running examples from the installation directory.

Example using the \ipopt solver:

\begin{lstlisting}
./bin/opflow -opflow_solver IPOPT -opflow_model POWER_BALANCE_POLAR -netfile $EXAGO_DIR/datafiles/case9/case9mod.m -print_output
[ExaGO] Creating OPFlow


******************************************************************************
This program contains Ipopt, a library for large-scale nonlinear optimization.
 Ipopt is released as open source code under the Eclipse Public License (EPL).
         For more information visit http://projects.coin-or.org/Ipopt
******************************************************************************

This is Ipopt version 3.12.10, running with linear solver ma27.

Number of nonzeros in equality constraint Jacobian...:      114
Number of nonzeros in inequality constraint Jacobian.:       72
Number of nonzeros in Lagrangian Hessian.............:       96

Total number of variables............................:       24
                     variables with only lower bounds:        0
                variables with lower and upper bounds:       16
                     variables with only upper bounds:        0
Total number of equality constraints.................:       18
Total number of inequality constraints...............:       18
        inequality constraints with only lower bounds:        0
   inequality constraints with lower and upper bounds:       18
        inequality constraints with only upper bounds:        0

iter    objective    inf_pr   inf_du lg(mu)  ||d||  lg(rg) alpha_du alpha_pr  ls
   0  1.0318125e+04 1.80e+00 1.00e+02  -1.0 0.00e+00    -  0.00e+00 0.00e+00   0
   1  7.7157691e+03 1.17e+00 1.03e+02  -1.0 1.08e+00    -  6.27e-01 3.50e-01f  1
   2  7.6608235e+03 1.15e+00 1.01e+02  -1.0 6.28e+00    -  1.20e-02 1.37e-02f  1
   3  7.4466686e+03 1.09e+00 3.06e+02  -1.0 3.74e+00    -  4.15e-03 5.81e-02f  1
   4  5.4292675e+03 3.92e-01 4.83e+03  -1.0 7.34e-01    -  3.34e-03 6.40e-01f  1
   5  4.5792834e+03 2.24e-01 1.51e+03  -1.0 6.46e-01   2.0 8.77e-03 7.37e-01f  1
   6  4.2907579e+03 1.20e-02 3.57e+02  -1.0 3.36e-01    -  5.37e-01 1.00e+00f  1
   7  4.1690456e+03 4.40e-02 5.31e+01  -1.0 3.31e-01    -  9.22e-01 1.00e+00f  1
   8  4.1687926e+03 4.88e-04 1.93e+00  -1.0 4.79e-02   1.5 1.00e+00 1.00e+00h  1
   9  4.1497176e+03 1.19e-02 9.92e+00  -2.5 1.87e-01    -  8.38e-01 1.00e+00f  1
iter    objective    inf_pr   inf_du lg(mu)  ||d||  lg(rg) alpha_du alpha_pr  ls
  10  4.1463942e+03 1.09e-02 5.09e-01  -2.5 1.15e-01    -  8.71e-01 1.00e+00h  1
  11  4.1449657e+03 1.47e-03 1.79e-02  -2.5 2.75e-02    -  1.00e+00 1.00e+00h  1
  12  4.1445415e+03 6.63e-04 9.12e-02  -3.8 1.48e-02    -  1.00e+00 6.30e-01h  1
  13  4.1444705e+03 3.43e-04 4.96e-02  -3.8 2.08e-02    -  1.00e+00 8.93e-01h  1
  14  4.1444809e+03 4.48e-05 1.79e-04  -3.8 6.82e-03    -  1.00e+00 1.00e+00f  1
  15  4.1444611e+03 1.96e-05 4.55e-03  -5.7 4.57e-03    -  1.00e+00 9.34e-01h  1
  16  4.1444607e+03 6.49e-06 1.17e-05  -5.7 2.60e-03    -  1.00e+00 1.00e+00h  1
  17  4.1444605e+03 1.19e-06 2.07e-06  -7.0 1.11e-03    -  1.00e+00 1.00e+00h  1
  18  4.1444605e+03 1.58e-07 3.29e-07  -7.0 4.06e-04    -  1.00e+00 1.00e+00h  1

Number of Iterations....: 18

                                   (scaled)                 (unscaled)
Objective...............:   9.2925122354655841e+01    4.1444604570176507e+03
Dual infeasibility......:   3.2927387965389691e-07    1.4685615032563802e-05
Constraint violation....:   2.6639677713768961e-08    2.6639677713768961e-08
Complementarity.........:   4.7840038622596930e-07    2.1336657225678232e-05
Overall NLP error.......:   4.7840038622596930e-07    2.1336657225678232e-05


Number of objective function evaluations             = 19
Number of objective gradient evaluations             = 19
Number of equality constraint evaluations            = 19
Number of inequality constraint evaluations          = 19
Number of equality constraint Jacobian evaluations   = 19
Number of inequality constraint Jacobian evaluations = 19
Number of Lagrangian Hessian evaluations             = 18
Total CPU secs in IPOPT (w/o function evaluations)   =      0.025
Total CPU secs in NLP function evaluations           =      0.002

EXIT: Optimal Solution Found.
=============================================================
Optimal Power Flow
=============================================================
Model                               POWER_BALANCE_POLAR
Solver                              IPOPT
Objective                           MIN_GEN_COST
Initialization                      MIDPOINT
Gen. bus voltage mode               VARIABLE_WITHIN_BOUNDS
Load loss allowed                   NO
Power imbalance allowed             NO
Ignore line flow constraints        NO

Number of variables                 24
Number of equality constraints      18
Number of inequality constraints    18

Convergence status                  CONVERGED
Objective value                     4144.46

----------------------------------------------------------------------
Bus        Pd      Qd      Vm      Va      mult_Pmis      mult_Qmis      Pslack         Qslack        
----------------------------------------------------------------------
1         0.00    0.00   1.100   0.000      2102.91         0.00         0.00         0.00
2         0.00    0.00   1.095   3.928      2059.18        -0.00         0.00         0.00
3         0.00    0.00   1.087   2.120      2065.15        -0.00         0.00         0.00
4         0.00    0.00   1.097  -1.993      2103.16         0.08         0.00         0.00
5        75.00   50.00   1.079  -3.060      2113.45         7.29         0.00         0.00
6        90.00   30.00   1.087  -3.927      2129.85         1.62         0.00         0.00
7         0.00    0.00   1.100   0.535      2059.57        -0.04         0.00         0.00
8       100.00   35.00   1.089  -1.720      2079.34         2.99         0.00         0.00
9         0.00    0.00   1.100  -0.135      2065.43        -0.09         0.00         0.00

----------------------------------------------------------------------------------------
From       To       Status     Sft      Stf     Slim     mult_Sf  mult_St 
----------------------------------------------------------------------------------------
1          4          1       73.18    72.98   380.00    -0.00    -0.00
2          7          1      114.18   114.68   250.00    -0.00    -0.00
3          9          1       83.57    84.60   300.00    -0.00    -0.00
4          5          1       29.68    40.50   250.00    -0.00    -0.00
4          6          1       44.86    46.03   250.00    -0.00    -0.00
5          7          1       51.29    49.04   250.00    -0.00    -0.00
6          9          1       48.94    51.43   150.00    -0.00    -0.00
7          8          1       66.61    68.11   250.00    -0.00    -0.00
8          9          1       38.86    34.15   150.00    -0.00    -0.00

----------------------------------------------------------------------------------------
Gen      Status     Fuel     Pg       Qg       Pmin     Pmax     Qmin     Qmax  
----------------------------------------------------------------------------------------
1          1    UNDEFINED    72.86     6.79    10.00   350.00  -300.00   300.00
2          1    UNDEFINED   114.07    -5.13    10.00   300.00  -300.00   300.00
3          1    UNDEFINED    80.21   -23.47    10.00   270.00  -300.00   300.00
[ExaGO] Finalizing opflow application.
\end{lstlisting}

Example using the \hiop solver on the CPU with ACPF initialization:

\begin{lstlisting}
./bin/opflow -opflow_solver HIOP -opflow_model POWER_BALANCE_HIOP -netfile $EXAGO_DIR/datafiles/case9/case9mod.m -opflow_initialization ACPF -hiop_compute CPU -hiop_verbosity_level 3 -print_output
[ExaGO] Creating OPFlow

[Warning] Detected 1 fixed variables out of a total of 24.
===============
Hiop SOLVER
===============
Using 1 MPI ranks.
---------------
Problem Summary
---------------
Total number of variables: 24
     lower/upper/lower_and_upper bounds: 16 / 16 / 16
Total number of equality constraints: 18
Total number of inequality constraints: 18
     lower/upper/lower_and_upper bounds: 18 / 18 / 18
iter    objective     inf_pr     inf_du   lg(mu)  alpha_du   alpha_pr linesrch
   0  4.6670737e+03 3.714e-11  2.891e+03  -1.00  0.000e+00  0.000e+00  -(-)
   1  4.6431205e+03 3.373e-04  2.791e+03  -1.00  3.013e-01  3.498e-02  1(f)
   2  4.4461241e+03 6.595e-02  2.105e+03  -1.00  4.910e-01  2.467e-01  1(s)
   3  4.1942969e+03 6.464e-02  5.808e+02  -1.00  5.986e-01  8.573e-01  1(s)
   4  4.1615434e+03 8.234e-03  1.159e+02  -1.00  7.556e-01  1.000e+00  1(s)
   5  4.1471588e+03 4.414e-03  1.291e+02  -1.00  1.000e+00  1.000e+00  1(s)
   6  4.1441777e+03 2.959e-02  7.889e+01  -1.00  4.818e-01  4.030e-01  1(s)
   7  4.1448270e+03 7.926e-03  4.949e+01  -1.00  1.000e+00  1.000e+00  1(s)
   8  4.1448562e+03 1.506e-03  2.576e-01  -1.00  1.000e+00  1.000e+00  1(s)
   9  4.1444966e+03 3.576e-04  2.859e+00  -3.82  9.631e-01  6.968e-01  1(s)
iter    objective     inf_pr     inf_du   lg(mu)  alpha_du   alpha_pr linesrch
  10  4.1444617e+03 1.671e-04  2.747e+00  -3.82  9.520e-01  7.537e-01  1(s)
  11  4.1444597e+03 5.733e-05  1.190e-01  -3.82  9.716e-01  1.000e+00  1(s)
  12  4.1444610e+03 6.929e-06  5.005e-04  -3.82  1.000e+00  1.000e+00  1(s)
  13  4.1444605e+03 1.681e-06  1.400e-04  -5.73  1.000e+00  1.000e+00  1(h)
  14  4.1444605e+03 2.574e-07  2.377e-05  -5.73  1.000e+00  1.000e+00  1(h)
  15  4.1444604e+03 1.411e-08  1.319e-06  -5.73  1.000e+00  1.000e+00  1(h)
  16  4.1444604e+03 2.396e-10  2.047e-08  -7.00  1.000e+00  1.000e+00  1(h)
Successfull termination.
Total time 0.933 sec 
Hiop internal time:     total 0.932 sec     avg iter 0.058 sec 
    internal total std dev across ranks 0.000 percent
Fcn/deriv time:     total=0.001 sec  ( obj=0.000 grad=0.000 cons=0.000 Jac=0.000 Hess=0.001) 
    Fcn/deriv total std dev across ranks 0.000 percent
Fcn/deriv #: obj 18 grad 18 eq cons 18 ineq cons 18 eq Jac 18 ineq Jac 18
Total KKT time 0.931 sec 
update init 0.848sec     update linsys 0.000 sec     fact 0.071 sec 
solve rhs-manip 0.000 sec     triangular solve 0.012 sec 

=============================================================
Optimal Power Flow
=============================================================
Model                               POWER_BALANCE_HIOP
Solver                              HIOP
Objective                           MIN_GEN_COST
Initialization                      ACPF
Gen. bus voltage mode               VARIABLE_WITHIN_BOUNDS
Load loss allowed                   NO
Power imbalance allowed             NO
Ignore line flow constraints        NO

Number of variables                 24
Number of equality constraints      18
Number of inequality constraints    18

Convergence status                  CONVERGED
Objective value                     4144.46

----------------------------------------------------------------------
Bus        Pd      Qd      Vm      Va      mult_Pmis      mult_Qmis      Pslack         Qslack        
----------------------------------------------------------------------
1         0.00    0.00   1.100  -0.000      2102.91         0.00         0.00         0.00
2         0.00    0.00   1.095   3.928      2059.18        -0.00         0.00         0.00
3         0.00    0.00   1.087   2.120      2065.15        -0.00         0.00         0.00
4         0.00    0.00   1.097  -1.993      2103.17         0.07         0.00         0.00
5        75.00   50.00   1.079  -3.060      2113.46         7.29         0.00         0.00
6        90.00   30.00   1.087  -3.927      2129.85         1.62         0.00         0.00
7         0.00    0.00   1.100   0.535      2059.57        -0.04         0.00         0.00
8       100.00   35.00   1.089  -1.720      2079.34         2.99         0.00         0.00
9         0.00    0.00   1.100  -0.135      2065.43        -0.09         0.00         0.00

----------------------------------------------------------------------------------------
From       To       Status     Sft      Stf     Slim     mult_Sf  mult_St 
----------------------------------------------------------------------------------------
1          4          1       73.18    72.98   380.00    -0.00    -0.00
2          7          1      114.18   114.68   250.00    -0.00    -0.00
3          9          1       83.58    84.61   300.00    -0.00    -0.00
4          5          1       29.69    40.50   250.00    -0.00    -0.00
4          6          1       44.86    46.03   250.00    -0.00    -0.00
5          7          1       51.29    49.04   250.00    -0.00    -0.00
6          9          1       48.94    51.43   150.00    -0.00    -0.00
7          8          1       66.61    68.11   250.00    -0.00    -0.00
8          9          1       38.87    34.15   150.00    -0.00    -0.00

----------------------------------------------------------------------------------------
Gen      Status     Fuel     Pg       Qg       Pmin     Pmax     Qmin     Qmax  
----------------------------------------------------------------------------------------
1          1    UNDEFINED    72.86     6.80    10.00   350.00  -300.00   300.00
2          1    UNDEFINED   114.07    -5.13    10.00   300.00  -300.00   300.00
3          1    UNDEFINED    80.21   -23.48    10.00   270.00  -300.00   300.00
[ExaGO] Finalizing opflow application.
\end{lstlisting}

Example with HIOP solver on GPU with load loss activated. In this example, the load at bus 5 is
increased to 750 MW leading to an infeasible power flow. Activating the load loss causes shedding
of load at bus 5 and as a result makes the optimization converge.

\begin{lstlisting}
./opflow -opflow_solver HIOP -opflow_model PBPOLRAJAHIOP -netfile $EXAGO_DIR/datafiles/case9/case9mod_loadloss.m -hiop_compute_mode GPU -hiop_verbosity_level 3 -print_output -opflow_include_loadloss_variables
[ExaGO] Creating OPFlow

[Warning] Detected 1 fixed variables out of a total of 30.
===============
Hiop SOLVER
===============
Using 1 MPI ranks.
---------------
Problem Summary
---------------
Total number of variables: 30
     lower/upper/lower_and_upper bounds: 22 / 22 / 22
Total number of equality constraints: 18
Total number of inequality constraints: 18
     lower/upper/lower_and_upper bounds: 18 / 18 / 18
iter    objective     inf_pr     inf_du   lg(mu)  alpha_du   alpha_pr linesrch
   0  1.4867531e+04 7.490e+00  1.000e+05  -1.00  0.000e+00  0.000e+00  -(-)
   1  1.3403771e+04 7.490e+00  1.000e+05  -1.00  2.729e-04  1.702e-05  1(s)
   2  1.1742043e+04 7.489e+00  9.993e+04  -1.00  5.655e-03  1.102e-04  1(s)
   3  1.0453140e+04 7.484e+00  8.993e+04  -1.00  1.258e-01  6.948e-04  1(s)
   4  1.0933710e+04 7.247e+00  8.722e+04  -1.00  3.011e-02  3.073e-02  1(s)
   5  1.2262728e+04 6.691e+00  1.217e+05  -1.00  3.642e-04  7.537e-02  1(s)
   6  1.2718100e+04 6.509e+00  1.303e+05  -1.00  2.837e-04  2.663e-02  1(s)
   7  1.2721404e+04 6.508e+00  1.302e+05  -1.00  1.583e-02  1.930e-04  1(s)
   8  1.6122803e+04 5.696e+00  7.620e+04  -1.00  1.954e-03  1.492e-01  1(S)
   9  1.6366303e+04 5.692e+00  7.616e+04  -1.00  9.924e-02  8.075e-04  1(s)
iter    objective     inf_pr     inf_du   lg(mu)  alpha_du   alpha_pr linesrch
  10  1.7501119e+04 5.671e+00  7.587e+04  -1.00  3.082e-03  3.708e-03  1(s)
  11  1.7624878e+04 5.668e+00  8.682e+04  -1.00  9.630e-02  3.944e-04  1(s)
  12  2.2838548e+04 5.575e+00  8.354e+04  -1.00  1.071e-03  1.646e-02  1(s)
  ...
  ...
  47  7.4002786e+05 5.392e-03  5.252e+05  -1.00  1.365e-04  4.401e-03  1(s)
  48  7.4002761e+05 5.391e-03  5.332e+05  -1.00  2.888e-02  1.880e-03  1(s)
  49  7.4003537e+05 3.310e-03  4.892e+05  -1.00  3.124e-03  1.000e+00  1(s)
iter    objective     inf_pr     inf_du   lg(mu)  alpha_du   alpha_pr linesrch
  50  7.4002879e+05 3.234e-03  3.147e+05  -1.00  1.000e+00  2.273e-02  1(s)
  51  7.4003883e+05 1.340e-05  2.982e+03  -1.00  6.601e-01  1.000e+00  1(s)
  52  7.4003471e+05 3.828e-04  1.015e+01  -1.00  1.000e+00  1.000e+00  1(f)
  53  7.4003556e+05 2.461e-08  2.164e-01  -3.82  9.883e-01  1.000e+00  1(h)
  54  7.4003556e+05 2.665e-14  5.489e-07  -5.73  1.000e+00  1.000e+00  1(f)
Successfull termination.
Total time 1.527 sec 
Hiop internal time:     total 1.465 sec     avg iter 0.027 sec 
    internal total std dev across ranks 0.000 percent
Fcn/deriv time:     total=0.052 sec  ( obj=0.009 grad=0.003 cons=0.012 Jac=0.011 Hess=0.018) 
    Fcn/deriv total std dev across ranks 0.000 percent
Fcn/deriv #: obj 97 grad 56 eq cons 97 ineq cons 97 eq Jac 56 ineq Jac 56
Total KKT time 1.249 sec 
update init 0.773sec     update linsys 0.029 sec     fact 0.237 sec 
solve rhs-manip 0.172 sec     triangular solve 0.038 sec 

=============================================================
Optimal Power Flow
=============================================================
Model                               PBPOLRAJAHIOP
Solver                              HIOP
Objective                           MIN_GEN_COST
Initialization                      MIDPOINT
Gen. bus voltage mode               VARIABLE_WITHIN_BOUNDS
Load loss allowed                   YES
Load loss penalty ($)               1000.
Power imbalance allowed             NO
Ignore line flow constraints        NO

Number of variables                 30
Number of equality constraints      18
Number of inequality constraints    18

Convergence status                  CONVERGED
Objective value                     740035.56

----------------------------------------------------------------------
Bus        Pd      Qd      Vm      Va      mult_Pmis      mult_Qmis      Pslack         Qslack        
----------------------------------------------------------------------
1         0.00    0.00   1.100   0.000      4182.05         0.00         0.00         0.00
2         0.00    0.00   1.100  12.398      4028.36         0.00         0.00         0.00
3         0.00    0.00   1.100  16.811      6041.35         0.00         0.00         0.00
4         0.00    0.00   1.021  -4.918      5930.37     19899.38         0.00         0.00
5       417.50  109.68   0.900 -15.480    100000.00    100000.00         0.00         0.00
6        90.00   30.00   1.018  -2.414      7827.64     16824.43         0.00         0.00
7         0.00    0.00   1.052   5.271     19366.68     29169.45         0.00         0.00
8       100.00   35.00   1.053   5.150     15594.67     21709.27         0.00         0.00
9         0.00    0.00   1.083   9.964      7193.17      9437.04         0.00         0.00

----------------------------------------------------------------------------------------
From       To       Status     Sft      Stf     Slim     mult_Sf  mult_St 
----------------------------------------------------------------------------------------
1          4          1      230.52   213.89   380.00    -0.00    -0.00
2          7          1      250.00   239.12   250.00  2404.19     0.00
3          9          1      246.68   242.96   300.00     0.00     0.00
4          5          1      250.00   227.52   250.00 14113.38     0.00
4          6          1       47.20    50.98   250.00    -0.00    -0.00
5          7          1      210.17   239.03   250.00     0.00     0.00
6          9          1      137.83   144.52   150.00     0.00     0.00
7          8          1       10.10     7.53   250.00    -0.00    -0.00
8          9          1      100.92    98.82   150.00    -0.00    -0.00

----------------------------------------------------------------------------------------
Gen      Status     Fuel     Pg       Qg       Pmin     Pmax     Qmin     Qmax  
----------------------------------------------------------------------------------------
1          1    UNDEFINED   167.37   158.52    10.00   350.00  -300.00   300.00
2          1    UNDEFINED   229.90    98.21    10.00   300.00  -300.00   300.00
3          1    UNDEFINED   242.50    45.19    10.00   270.00  -300.00   300.00
[ExaGO] Finalizing opflow application.
\end{lstlisting}


\chapter{Multi-period optimal power flow (TCOPFLOW)}\label{chap:tcopflow}

TCOPFLOW solves a full AC multi-period optimal power flow problem with the objective of minimizing the total cost over the given time horizon while adhering to constraints for each period and between consecutive time-periods (ramping constraints). 

\section{Formulation}
The multi-period optimal power flow problem is a series of optimal power flow problems coupled via temporal constraints. The generator real power deviation ($p_{jt}^{\text{g}} - p_{jt-\Delta{t}}^{\text{g}}$) constrained within the ramp limits form the temporal constraints. An illustration of the temporal constraints is shown in Fig. \ref{fig:tcopflow} with four time steps. Each time-step $t$ is coupled with its preceding time $t-\Delta{t}$, where $\Delta{t}$ is the time-step where the objective is to find a least cost dispatch for the given time horizon.

\definecolor{lavander}{cmyk}{0,0.48,0,0}
\definecolor{violet}{cmyk}{0.79,0.88,0,0}
\definecolor{burntorange}{cmyk}{0,0.52,1,0}

\def\lav{lavander!90}
\def\oran{orange!30}

\tikzstyle{time}=[draw,circle,violet,bottom color=\lav,
                  top color= white, text=violet,minimum width=20pt]
\tikzstyle{base}=[draw,circle,burntorange, left color=\oran,
                       text=violet,minimum width=20pt]

\begin{figure}[h!]
\centering
\begin{tikzpicture}[auto, thick]
  \node[time,label=below:$t_0$] (t0) at (0,0) {};
  \node[time,label=below:$t_1$] (t1) at (2,0) {};
  \node[time,label=below:$t_2$] (t2) at (4,0) {};
  \node[time,label=below:$t_3$] (t3) at (6,0) {};
  
  \path (t0) edge (t1);
  \path (t1) edge (t2);
  \path (t2) edge (t3);


\end{tikzpicture}
\caption{Multi-period optimal power flow example with four time-steps. The lines connecting the different time-periods denote the coupling between them.}
\label{fig:tcopflow}
\end{figure}


In general form, the equations for multi-period optimal power flow are given by (\ref{eq:tcopflow_start}) -- (\ref{eq:tcopflow_end}). TCOPFLOW solves to minimize the total generation cost $\sum_{t=0}^{N_t-1}f(x_t)$ over the time horizon, where $N_t$ is the number of time-steps. At each time-step, the equality constraints ($g(x_t)$), inequality $h(x_t)$, and the lower/upper limit ($x^-$,$x^+$) constraints need to be satisfied. Equation (\ref{eq:tcopflow_end}) represents the coupling between the consecutive time-steps. It is the most common form of coupling that limits the deviation of the real power generation at time $t$ from its preceding time-step $t-\Delta{t}$ to within its ramping capability $\delta_t{x}$.


\begin{align}
\centering
\text{min}&~\sum_{t=0}^{N_t-1} f(x_t) &  \label{eq:tcopflow_start}\\
&\text{s.t.}& \nonumber \\
&~g(x_t) = 0,                                        &t \in \left[0,N_t-1\right]& \\
&~h(x_t) \le 0,                                      &t \in \left[0,N_t-1\right]& \\
x^- & \le x_t \le x^+,                               &t\in \left[0,N_t-1\right]& \\
-\delta_t{x} & \le x_t - x_{t-\Delta{t}} \le \delta_t{x},&t \in \left[1,N_t-1\right]&
\label{eq:tcopflow_end}
\end{align}

\section{Solvers}\label{sec:tcopflow_solvers}%
\tcopflow solves the coupled multi-period using the solver \ipopt. Currently, \exago only supports solving \tcopflow using \ipopt~, which is also the default solver. Solving TCOPFLOW is supported only on a single process. %Option:\\ %\opflowoption{-tcopflow_solver IPOPT}

\section{Input and Output}
\begin{itemize}
    \item \textbf{Network file:} The network file describing the network details. Only \matpower format files are currently supported.
    \item \textbf{Load data:} One file for load real power and one for reactive power. The files need to be in CSV format. An example of the format for the 9-bus case is \href{https://gitlab.pnnl.gov/exasgd/frameworks/exago/-/tree/master/datafiles/case9}{here}.
    \item \textbf{Wind generation:} The wind generation time-series described in CSV format. See an example of the format \href{https://gitlab.pnnl.gov/exasgd/frameworks/exago/-/tree/master/datafiles/case9}{here}.
\end{itemize}
If the load data and/or wind generation profiles are not set then a flat profile is assumed, i.e., the load and wind generation for all hours is constant.

The \tcopflow output is saved to a directory named \emph{tcopflowout}. This directory contains $N_t$ files, one for each time-step, in \matpower data file format.

\section{Usage}
\begin{lstlisting}
./tcopflow -netfile <netfilename> -tcopflow_pload_profile <ploadprofile> \
-tcopflow_windgen_profile <windgenprofile> <tcopflowoptions> \
-tcopflow_qload_profile <qloadprofile>
\end{lstlisting}
\section{Options}
See table \ref{tab:tcopflow_options}. In addition, all \opflow options given in Table \ref{tab:opflow_options} can be used.
\begin{table}[!htbp]
  \caption{TCOPFLOW options}
  \small
  \begin{tabular}{|p{0.4\textwidth}|p{0.25\textwidth}|p{0.25\textwidth}|}
    \hline
    \textbf{Option} & \textbf{Meaning} & \textbf{Values (Default value)} \\ \hline
    -netfile & Network file name & string $<$ 4096 characters (\href{https://gitlab.pnnl.gov/exasgd/frameworks/exago/-/blob/master/datafiles/case9/case9mod_gen3_wind.m}{case9mod\_gen3\_wind.m}) \\ \hline
    -save\_output & Save output to file & 0 or 1 (0) \\ \hline
    -tcopflow\_pload\_profile & Real power load profile & string $<$ 4096 characters (\href{https://gitlab.pnnl.gov/exasgd/frameworks/exago/-/blob/master/datafiles/case9/load_P.csv}{load\_P.csv}) \\ \hline
    -tcopflow\_qload\_profile & Reactive power load profile & string $<$ 4096 characters (\href{https://gitlab.pnnl.gov/exasgd/frameworks/exago/-/blob/master/datafiles/case9/load_Q.csv}{load\_Q.csv}) \\ \hline
    -tcopflow\_windgen\_profile & Wind generation profile & string $<$ 4096 characters (\href{https://gitlab.pnnl.gov/exasgd/frameworks/exago/-/blob/master/datafiles/case9/scenarios_9bus.csv}{case9/scenarios\_9bus.csv}) \\ \hline
    -tcopflow\_dT & Length of time-step (minutes) & double (5.0) \\ \hline
    -tcopflow\_duration & Total duration (hours) & double (0.5) \\ \hline 
    -tcopflow\_iscoupling & Coupling between time steps (ramp constraints) & 0 or 1 (0) \\ \hline
  \end{tabular}
  \label{tab:tcopflow_options}
\end{table}
\section{Examples}
Some \tcopflow example runs are provided with some sample output. Options are the default options provided in table \ref{tab:opflow_options} and \ref{tab:tcopflow_options} unless otherwise specified. Sample output is generated by running examples from the installation directory.

Example using \ipopt solver:

\begin{lstlisting}
$ ./bin/tcopflow -tcopflow_iscoupling 1 -print_output
[ExaGO INFO]: -- Checking ... -options_file not passed exists: no
TCOPFLOW: Application created
TCOPFLOW: Duration = 0.500000 hours, timestep = 5.000000 minutes, \
                                     number of time-steps = 7
TCOPFLOW: Using IPOPT solver
TCOPFLOW: Setup completed
Setting: "0" is not a valid setting for Option: tol. 
Check the option documentation.

### tol (Real Number) ###
Category: Convergence
Description: Desired convergence tolerance (relative).
0 < (1e-08) <= +inf

*************************************************************************
This program contains Ipopt...
*************************************************************************

This is Ipopt version 3.12.10, running with linear solver ma27.

Number of nonzeros in equality constraint Jacobian...:      770
Number of nonzeros in inequality constraint Jacobian.:      610
Number of nonzeros in Lagrangian Hessian.............:      630

Total number of variables............................:      161
                     variables with only lower bounds:        0
                variables with lower and upper bounds:      105
                     variables with only upper bounds:        0
Total number of equality constraints.................:      126
Total number of inequality constraints...............:      186
        inequality constraints with only lower bounds:       42
   inequality constraints with lower and upper bounds:      144
        inequality constraints with only upper bounds:        0

iter  objective  inf_pr  inf_du lg(mu) ||d|| lg(rg) alpha_du alpha_pr  ls
0  5.2094875e+04 1.80e+00 5.02e+01  -1.0 0.00e+00 - 0.00e+00 0.00e+00   0
1  3.6277420e+04 1.40e+00 1.06e+02  -1.0 2.05e+00 - 1.88e-01 2.20e-01f  1
...
33  2.0843675e+04 1.13e-14 3.32e-09  -8.6 1.76e-08 - 1.00e+00 1.00e+00h 1

Number of Iterations....: 33

                                   (scaled)                 (unscaled)
Objective...........:   4.6734697780313832e+02    2.0843675210019970e+04
Dual infeasibility..:   3.3166774138882895e-09    1.4792381265941772e-07
Constraint violation:   1.1296519275560968e-14    1.1296519275560968e-14
Complementarity.....:   4.9900570925632554e-09    2.2255654632832121e-07
Overall NLP error...:   4.9900570925632554e-09    2.2255654632832121e-07


Number of objective function evaluations             = 47
Number of objective gradient evaluations             = 34
Number of equality constraint evaluations            = 47
Number of inequality constraint evaluations          = 47
Number of equality constraint Jacobian evaluations   = 34
Number of inequality constraint Jacobian evaluations = 34
Number of Lagrangian Hessian evaluations             = 33
Total CPU secs in IPOPT (w/o function evaluations)   =      0.025
Total CPU secs in NLP function evaluations           =      0.032

EXIT: Optimal Solution Found.
=============================================================
        Multi-Period Optimal Power Flow
=============================================================
OPFLOW Model                        POWER_BALANCE_POLAR
Solver                              IPOPT
Duration (minutes)                  30.00
Time-step (minutes)                 5.00
Number of steps                     7
Active power demand profile         datafiles/case9/load_P.csv
Rective power demand profile        datafiles/case9/load_Q.csv
Wind generation profile             datafiles/case9/scenarios_9bus.csv
Load loss allowed                   NO
Power imbalance allowed             NO
Ignore line flow constraints        NO

Number of variables                 168
Number of equality constraints      126
Number of inequality constraints    168
Number of coupling constraints      18

Convergence status                  CONVERGED
Objective value                     20843.68

----------------------------------------------------------------------
Bus        Pd      Qd      Vm      Va      mult_Pmis      mult_Qmis
----------------------------------------------------------------------
1         0.00    0.00   1.040   0.000      2063.35        -0.00
2         0.00    0.00   1.025   4.553      2013.40         0.00
3         0.00    0.00   1.025   2.955      2017.53         0.00
4         0.00    0.00   1.037  -2.174      2063.59        -0.74
5        75.00   30.00   1.025  -3.405      2074.28         2.68
6        90.00   30.00   1.023  -4.213      2092.22         4.26
7         0.00    0.00   1.033   0.783      2014.09         3.85
8       100.00   35.00   1.022  -1.667      2035.30         7.65
9         0.00    0.00   1.036   0.264      2018.00         3.11

--------------------------------------------------------------------------
From       To       Status     Sft      Stf     Slim     mult_Sf  mult_St
--------------------------------------------------------------------------
1          4          1       71.31    71.13   380.00    -0.00    -0.00
2          7          1      111.80   112.67   250.00    -0.00    -0.00
3          9          1       86.66    87.56   300.00    -0.00    -0.00
4          5          1       28.34    34.62   250.00    -0.00    -0.00
4          6          1       42.81    45.47   250.00    -0.00    -0.00
5          7          1       47.86    51.21   250.00    -0.00    -0.00
6          9          1       49.48    52.48   150.00    -0.00    -0.00
7          8          1       63.89    65.25   250.00    -0.00    -0.00
8          9          1       41.64    36.64   150.00    -0.00    -0.00

------------------------------------------------------------------------
Gen Status   Fuel     Pg       Qg       Pmin     Pmax     Qmin     Qmax
------------------------------------------------------------------------
1     1      COAL    71.06     5.89    10.00   350.00  -300.00   300.00
2     1      COAL   111.38    -9.71    10.00   300.00  -300.00   300.00
3     1      WIND    85.00   -16.87     0.00    85.00  -300.00   300.00
[ExaGO INFO]: Finalizing tcopflow application.
\end{lstlisting}



\chapter{Security-constrained optimal power flow (\scopflow)}\label{chap:scopflow}
SCOPFLOW solves a contingency-constrained optimal power flow problem. The problem is set up as a two-stage optimization problem where the first-stage (base-case) represents the normal operation of the grid and the second-stage comprises $N_c$ contingency scenarios. Each contingency scenario can be single or multi-period.

\section{Formulation}

\subsection{Single-period}

The contingency-constrained optimal power flow (popularly termed as security-constrained optimal power flow (SCOPF) in power system parlance) attempts to find a least cost dispatch for the base case (or no contingency) while ensuring that if any of contingencies do occur then the system will be secure. This is illustrated in Fig. \ref{fig:scopflow} for a SCOPF with a base-case $c_0$ and three contingencies.

\definecolor{lavander}{cmyk}{0,0.48,0,0}
\definecolor{violet}{cmyk}{0.79,0.88,0,0}
\definecolor{burntorange}{cmyk}{0,0.52,1,0}

\def\lav{lavander!90}
\def\oran{orange!30}

\tikzstyle{contingency}=[draw,circle,violet,bottom color=\lav,
                  top color= white, text=violet,minimum width=20pt]
\tikzstyle{base}=[draw,circle,burntorange, left color=\oran,
                       text=violet,minimum width=20pt]
                       
\tikzstyle{cedge}=[color=red]

\begin{figure}[h!]
\centering
\begin{tikzpicture}[auto, thick]
  % Place base case
  \node[base,label=left:$c_0$] (base) at (0,0) {};
  
  \node[contingency,label=right:$c_1$] (c1) at (2,1) {};
  \node[contingency,label=right:$c_2$] (c2) at (2,0) {};
  \node[contingency,label=right:$c_3$] (c3) at (2,-1) {};
  
  \path[cedge] (base) edge (c1);
  \path[cedge] (base) edge (c2);
  \path[cedge] (base) edge (c3);
  
  
  
%  \foreach \place/\name in {{(0,-1)/a}, {(2,0)/b}, {(2,2)/c}, {(0,2)/d},
%           {(-2,0)/e}}
%    \node[superpeers] (\name) at \place {a};
%  \foreach \source/\dest in {a/b, a/c, a/d, b/c, c/d,a/e,d/e}
%    \path (\source) edge (\dest);
   %
   % Place normal peers
%  \foreach \pos/\i in {above left of/1, left of/2, below left of/3}
%    \node[peers, \pos = e] (e\i) {};
%   \foreach \speer/\peer in {e/e1,e/e2,e/e3}
%    \path (\speer) edge (\peer);
   %
%   \foreach \pos/\i in {above right of/1, right of/2, below right of/3}
%    \node[peers, \pos =b ] (b\i) {};
%   \foreach \speer/\peer in {b/b1,b/b2,b/b3}
%   \path (\speer) edge (\peer);
   %
%   \node[peers, above of=d] (d1){};
%   \path (d) edge (d1);
   %
%   \foreach \pos/\i in {below left of/1, below of/2}
%   \node[peers, \pos =a ] (a\i) {};
%   \foreach \speer/\peer in {a/a1,a/a2}
%   \path (\speer) edge (\peer);

\end{tikzpicture}
\caption{Contingency constrained optimal power flow example with three contingencies. $c_0$ represents the base case (or no contingency case). $c_1$, $c_2$, $c_3$ are the three contingency cases. Each of the contingency states is coupled with the base-case through ramping constraints (denoted by \textcolor{red}{red} lines)}
\label{fig:scopflow}
\end{figure}


In general form, the equations for contingency-constrained optimal power flow are given by
(\ref{eq:scopflow_start}) -- (\ref{eq:scopflow_end}). This is a two-stage
stochastic optimization problem where the first stage is the base case $c_0$ and
each of the contingency states $c_i, i \in [1,N_c]$ are second-stage
subproblems. SCOPFLOW aims to minimize the objective $\sum_{c=0}^{N_c}f(x_c)$,
while adhering to the equality $g_c(x_c)$, inequality $h_c(x_c)$, and the
lower/upper bound ($x^-$,$x^+$) constraints. Equation (\ref{eq:scopflow_end})
represents the coupling between the base-case $c_0$ and each of the contingency states
$c_i$. Equation (\ref{eq:scopflow_end}) is the most typical form of coupling
that limits the deviation of the contingency variables $x_c$ from the base $x_0$
to within $\Delta x_c$. An example of this constraint could be the allowed real power output deviation for the generators constrained by their ramp limit, which is currently the only constraint SCOPFLOW supports.


\begin{align}
\centering
\text{min}&~\sum_{c=0}^{N_c}f_c(x_c)&  \label{eq:scopflow_start}\\
&\text{s.t.}& \nonumber \\
&~g_c(x_c) = 0,                             &c \in \left[0,N_c\right]& \\
&~h_c(x_c) \le 0,                           &c \in \left[0,N_c\right]& \\
x^- & \le x_c \le x^+,                     &c\in \left[0,N_c\right]& \\
-\Delta x_c & \le x_c - x_0 \le \Delta x_c,&c \in \left[1,N_c\right]&
\label{eq:scopflow_end}
\end{align}

\subsection{Multiperiod}

In the multi-period version, each contingency is comprised of multiple time-periods. The multiple periods have variables and constraints as described in chapter \ref{chap:tcopflow}. An example of multi-contingency multi-period optimal power flow is illustrated in Fig. \ref{fig:ctopflow} for multi-period SCOPFLOW with three contingencies $c_1$, $c_2$, and $c_3$ coupled to the base case $c_0$. Each state is multi-period with two time-periods. Each time-step is coupled with its adjacent one through ramping constraints. We assume that the contingency is incident at the first time-step, i.e. at $t_0$. This results in the coupling between the contingency cases $c_i, i \in [1,N_c]$ and the base-case $c_0$ only at time-step $t_0$.

\definecolor{lavander}{cmyk}{0,0.48,0,0}
\definecolor{violet}{cmyk}{0.79,0.88,0,0}
\definecolor{burntorange}{cmyk}{0,0.52,1,0}

\def\lav{lavander!90}
\def\oran{orange!30}

\tikzstyle{time}=[draw,circle,violet,bottom color=\lav,
                  top color= white, text=violet,minimum width=20pt]
                  
\tikzstyle{base}=[draw,circle,burntorange, left color=\oran,
                       text=violet,minimum width=20pt]

\begin{figure}[h!]
\centering
\begin{tikzpicture}[auto, thick]
  \node[base,label=below:$t_0$] (c0t0) at (0,0) {};
  \node[time,label=below:$t_1$] (c0t1) at (2,0) {};
  \node[time,label=below:$t_2$] (c0t2) at (4,0) {};
  \node[time,label=below:$t_3$] (c0t3) at (6,0) {};
  
  \path (c0t0) edge (c0t1);
  \path (c0t1) edge (c0t2);
  \path (c0t2) edge (c0t3);
  
  % rectangle from (-1.0,-0.5) to (7.0,0.5)
   \node[draw,rectangle,dashed,minimum width=8cm,minimum height=1cm,label=left:{$\mathbf{c_0}$}] (s0) at (3.0,0.0) {};
  
  \node[time,label=below:$t_0$] (c1t0) at (0,2) {};
  \node[time,label=below:$t_1$] (c1t1) at (2,2) {};
  \node[time,label=below:$t_2$] (c1t2) at (4,2) {};
  \node[time,label=below:$t_3$] (c1t3) at (6,2) {};
  
  \path (c1t0) edge (c1t1);
  \path (c1t1) edge (c1t2);
  \path (c1t2) edge (c1t3);
  
  \path (c0t0) edge [bend left,color=red] (c1t0);
  
   % rectangle from (-1.0,1.5) to (7.0,2.5)
   \node[draw,rectangle,dashed,minimum width=8cm,minimum height=1cm,label=left:{$\mathbf{c_1}$}] (s0) at (3.0,2.0) {};
  
  
  
%  \foreach \place/\name in {{(0,-1)/a}, {(2,0)/b}, {(2,2)/c}, {(0,2)/d},
%           {(-2,0)/e}}
%    \node[superpeers] (\name) at \place {a};
%  \foreach \source/\dest in {a/b, a/c, a/d, b/c, c/d,a/e,d/e}
%    \path (\source) edge (\dest);
   %
   % Place normal peers
%  \foreach \pos/\i in {above left of/1, left of/2, below left of/3}
%    \node[peers, \pos = e] (e\i) {};
%   \foreach \speer/\peer in {e/e1,e/e2,e/e3}
%    \path (\speer) edge (\peer);
   %
%   \foreach \pos/\i in {above right of/1, right of/2, below right of/3}
%    \node[peers, \pos =b ] (b\i) {};
%   \foreach \speer/\peer in {b/b1,b/b2,b/b3}
%   \path (\speer) edge (\peer);
   %
%   \node[peers, above of=d] (d1){};
%   \path (d) edge (d1);
   %
%   \foreach \pos/\i in {below left of/1, below of/2}
%   \node[peers, \pos =a ] (a\i) {};
%   \foreach \speer/\peer in {a/a1,a/a2}
%   \path (\speer) edge (\peer);

\end{tikzpicture}
\caption{Multi-period contingency constrained optimal power flow example with two contingencies $c_0$ and $c_1$, each with four time-periods $t_0$, $t_1$, $t_2$, $t_3$. State $c_0,t_0$ represent the base case (no contingency) case. We assume that any contingency is incident at the first time-step, i.e., at $t_0$. The contingency states $c_1,t_0$ is coupled with the no-contingency state $c_0,t_0$ at time $t_0$. The {\textcolor{red}{red}} line denotes the coupling between the contingency.}
\label{fig:ctopflow}
\end{figure}

The overall objective of this contingency-constrained multi-period optimal power flow is to find a secure dispatch for base-case $c_0$ while adhering to contingency and temporal constraints. The general formulation of this problem is given in Eqs. (\ref{eq:ctopflow_start}) -- (\ref{eq:ctopflow_end}).

\begin{align}
\centering
\text{min}&~\sum_{c=0}^{N_c}\sum_{t=0}^{N_t-1}f_{ct}(x_{c,t})& \label{eq:ctopflow_start}\\
&\text{s.t.}& \nonumber \\
&~g_{ct}(x_{c,t}) = 0,                                        &c \in \left[0,N_c\right], t \in \left[0,N_t-1\right]& \\
&~h_{ct}(x_{c,t}) \le 0,                                      &c \in \left[0,N_c\right], t \in \left[0,N_t-1\right]& \\
x^- & \le x_{c,t} \le x^+,                               &c \in \left[0,N_c\right], t\in \left[0,N_t-1\right]& \\
-\Delta x_t & \le x_{c,t} - x_{c,t-\Delta{t}} \le \Delta x_t,&c \in \left[0,N_c\right], t \in \left[1,N_t-1\right]& \label{eq:ctopflow_time_coupling}\\
-\Delta x_c & \le x_{c,0} - x_{0,0} \le \Delta x_c,&c \in \left[1,N_c\right]& \\
\label{eq:ctopflow_end}
\end{align}

In this formulation, the objective is to reduce the cost for the base-case time
horizon, where $f(x_{0,t})$ is the objective cost for contingency $c_0$ at time
$t$. Equation (\ref{eq:ctopflow_end}) represents the coupling between the base
case $c_0$ and each contingency $c_i$ at time-step $t_0$. We use a simple box
constraint $\Delta x_c$ to restrict the  deviation of decision variables
$x_{c,0}$ from the base-case $x_{0,0}$. The bound $\Delta x_c$ could represent here, for example, the allowable reserve for each generator.

\section{Solvers}
\scopflow supports solving the optimization problem via \ipopt, \hiop, or \emph{EMPAR}. \ipopt can solve \scopflow on single rank only. \hiop supports solving the problem in parallel using a primal-decomposition algorithm. With HIOP, one can solve the subproblem either on the CPU or GPU by selecting the appropriate subproblem model and solver (see options table below).  However, note that \emph{ExaGO needs to be built with \ipopt even when using \hiop solver.} The \emph{EMPAR} solver does not solve the security-constrained ACOPF problem, it only solves the base-case and the contingencies independently with \opflow. It distributes the contingencies to different processes when executed in parallel.

\section{Input and Output}
To execute SCOPFLOW, the following files are required:
\begin{itemize}
    \item \textbf{Network file:} The network file describing the network details. Only \matpower format files are currently supported.
    \item \textbf{Contingency file:} The file describing the contingencies. Contingencies can be single or multiple outages. The contingency file needs to be described in PTI format.
\end{itemize}
If the multi-period option is chosen, then additional files describing the load and wind generation can be (optionally) set.
\begin{itemize}
    \item \textbf{Load data:} One file for load real power and one for reactive power. The files need to be in CSV format. An example of the format for the 9-bus case is \href{https://gitlab.pnnl.gov/exasgd/frameworks/exago/-/tree/master/datafiles/case9}{here}.
    \item \textbf{Wind generation:} The wind generation time-series described in CSV format. See an example of the format \href{https://gitlab.pnnl.gov/exasgd/frameworks/exago/-/tree/master/datafiles/case9}{here}.
\end{itemize}

The \scopflow output is saved to a directory named \texttt{scopflowout}. This
directory contains $N_c$ files to save the solution for each contingency in
MATPOWER datafile format. Each file has the name \texttt{cont_xx} where \texttt{xx} is the contingency number. 

If the multi-period option is chosen then $N_c$ subdirectories are created (one
for each contingency), and each subdirectory contains $N_t$ output files, one
for each time-period. The subdirectories have the naming convention
\texttt{cont_xx} and the output file are named as \texttt{t_yy} where \texttt{yy} is the time-step number.

\section{Usage}

\begin{lstlisting}
  ./scopflow -netfile <netfilename> -ctgcfile <ctgcfilename> \
  <scopflowoptions> [-scopflow_enable_multiperiod 1]
\end{lstlisting}

\section{Options}
See table \ref{tab:scopflow_options}. In addition, all \opflow options in Table \ref{tab:opflow_options} and \tcopflow options in Table \ref{tab:tcopflow_options} can be used.
\begin{table}[!htbp]
  \caption{SCOPFLOW options}
  \small
  \begin{tabular}{|p{0.3\textwidth}|p{0.2\textwidth}|p{0.3\textwidth}|p{0.2\textwidth}|}
    \hline
    \textbf{Option} & \textbf{Meaning} & \textbf{Values (Default value)} & \textbf{Compatibility} \\ \hline
    -netfile & Network file name & string $<$ 4096 characters (\href{https://gitlab.pnnl.gov/exasgd/frameworks/exago/-/blob/master/datafiles/case9/case9mod_gen3_wind.m}{case9mod\_gen3\_wind.m}) &\\ \hline
    -ctgcfile & Contingency file name & string $<$ 4096 characters (\href{https://gitlab.pnnl.gov/exasgd/frameworks/exago/-/blob/master/datafiles/case9/case9.cont}{case9.cont}) &\\ \hline
    -print\_output & Print output to screen & 0 or 1 (0) &\\ \hline
    -save\_output & Save output to directory & 0 or 1 (0) & Format determined by OPFLOW option. \\ \hline
    -scopflow\_output\_directory & Output directory path & ``scopflowout'' & \\ \hline
    -scopflow\_solver & Set solver for scopflow & IPOPT, HIOP, or EMPAR (IPOPT) &\\ \hline
    -scopflow\_subproblem\_solver & Set solver for subproblem & IPOPT or HIOP (IPOPT) &Only when using HIOP solver for SCOPFLOW \\ \hline
    -scopflow\_subproblem\_model & Set model for subproblem & See OPFLOW chapter &Only when using HIOP solver for SCOPFLOW \\ \hline
    -scopflow\_Nc & Number of contingencies & int (0. Passing -1 results in all contingencies in the file used) &\\ \hline
    -scopflow\_mode & Operation mode: Preventive or corrective & 0 or 1 (0) &\\ \hline
    -scopflow\_enable\_multiperiod & Multi-period SCOPFLOW & 0 or 1 (0) & IPOPT solver only\\ \hline
    -scopflow\_pload\_profile & Real power load profile & string (\href{https://gitlab.pnnl.gov/exasgd/frameworks/exago/-/blob/master/datafiles/case9/load_P.csv}{load\_P.csv}) &\\ \hline
    -scopflow\_qload\_profile & Reactive power load profile & string (\href{https://gitlab.pnnl.gov/exasgd/frameworks/exago/-/blob/master/datafiles/case9/load_Q.csv}{load\_Q.csv}) &\\ \hline
    -scopflow\_windgenprofile & Wind generation profile & string (\href{https://gitlab.pnnl.gov/exasgd/frameworks/exago/-/blob/master/datafiles/case9/scenarios_9bus.csv}{case9/scenarios\_9bus.csv}) &\\ \hline
    -scopflow\_dT & Length of time-step (minutes) & double (5.0) &\\ \hline
    -scopflow\_duration & Total duration (hours) & double (0.5) &\\ \hline 
  \end{tabular}
  \label{tab:scopflow_options}
\end{table}

Depending on the value chosen for \texttt{-scopflow_mode}, SCOPFLOW can operate
either in \emph{preventive} (mode = 0) or \emph{corrective} (mode = 1) mode. In
the preventive mode, the PV and PQ generator real power is fixed to its
corresponding base-case values. The generators at the reference bus pick up any
make-up power required for the contingency. The corrective mode allows deviation
of the PV and PQ generator real power from the base-case dispatch constrained by
its 30-min. ramp rate capability. The optimization decides the optimal
redispatch. One can have AGC control instead of having the generators proportionally share the deficit/excess power by using the option \emph(-opflow_use_agc).

The option \texttt{-scopflow_enable_multiperiod} 1 must be used in order to enable any of the options listed in table \ref{tab:scopflow_options} for multiperiod analysis.

\section{Examples}

Some \scopflow example runs are provided with some sample output. Options are the default options given in table \ref{tab:opflow_options}, \ref{tab:tcopflow_options} and \ref{tab:scopflow_options} unless otherwise specified. Sample output is generated by running examples in the installation directory.

Example using the \ipopt solver:

\begin{lstlisting}
./bin/scopflow -netfile $EXAGO_DIR/datafiles/case9/case9mod.m -ctgcfile $EXAGO_DIR/datafiles/case9/case9.cont -scopflow_Nc 4 -scopflow_solver IPOPT -print_output 
[ExaGO] SCOPFLOW: Application created
[ExaGO] SCOPFLOW running with 5 subproblems (base case + 4 contingencies)
[ExaGO] SCOPFLOW: Using IPOPT solver
[ExaGO] SCOPFLOW: Setup completed

******************************************************************************
This program contains Ipopt, a library for large-scale nonlinear optimization.
 Ipopt is released as open source code under the Eclipse Public License (EPL).
         For more information visit http://projects.coin-or.org/Ipopt
******************************************************************************

This is Ipopt version 3.12.10, running with linear solver ma27.

Number of nonzeros in equality constraint Jacobian...:      598
Number of nonzeros in inequality constraint Jacobian.:      328
Number of nonzeros in Lagrangian Hessian.............:      660

Total number of variables............................:      144
                     variables with only lower bounds:        0
                variables with lower and upper bounds:      104
                     variables with only upper bounds:        0
Total number of equality constraints.................:      114
Total number of inequality constraints...............:       82
        inequality constraints with only lower bounds:        0
   inequality constraints with lower and upper bounds:       82
        inequality constraints with only upper bounds:        0

iter    objective    inf_pr   inf_du lg(mu)  ||d||  lg(rg) alpha_du alpha_pr  ls
   0  1.0318125e+04 1.80e+00 1.00e+02  -1.0 0.00e+00    -  0.00e+00 0.00e+00   0
   1  9.2868867e+03 1.56e+00 8.62e+01  -1.0 1.08e+00    -  6.27e-01 1.31e-01f  1
   2  9.0292788e+03 1.50e+00 3.50e+02  -1.0 2.60e+00    -  1.37e-03 4.28e-02f  1
   3  6.8151659e+03 9.02e-01 3.07e+03  -1.0 1.03e+00    -  4.57e-02 4.09e-01f  1
   4  5.6911245e+03 5.63e-01 2.09e+03  -1.0 7.85e-01   2.0 1.22e-01 3.87e-01f  1
   5  5.6219674e+03 5.39e-01 2.00e+03  -1.0 8.34e-01    -  1.99e-01 4.29e-02f  1
   6  4.7820782e+03 2.34e-01 7.42e+02  -1.0 7.27e-01    -  5.45e-01 5.75e-01f  1
   7  4.1986838e+03 3.84e-02 2.59e+02  -1.7 4.11e-01    -  3.64e-01 1.00e+00f  1
   8  4.1593634e+03 6.26e-02 1.52e+01  -1.7 3.77e-01    -  5.76e-01 1.00e+00f  1
   9  4.1598457e+03 3.63e-04 1.53e+00  -2.5 4.57e-02   1.5 8.25e-01 1.00e+00h  1
iter    objective    inf_pr   inf_du lg(mu)  ||d||  lg(rg) alpha_du alpha_pr  ls
  10  4.1472241e+03 4.61e-02 5.44e+00  -3.8 1.15e-01    -  3.15e-01 1.00e+00f  1
  11  4.1437514e+03 3.32e-02 3.50e+00  -3.8 1.95e-01    -  5.75e-01 3.47e-01h  1
  12  4.1440398e+03 1.07e-02 1.64e+00  -3.8 7.68e-02    -  2.42e-01 6.18e-01h  1
  13  4.1444455e+03 1.33e-03 4.38e-02  -3.8 3.29e-02    -  9.49e-01 1.00e+00h  1
  14  4.1444954e+03 5.41e-06 2.57e-02  -3.8 2.31e-03   1.0 1.00e+00 1.00e+00h  1
  15  4.1444778e+03 1.47e-05 1.41e-02  -3.8 3.80e-03   0.6 1.00e+00 1.00e+00h  1
  16  4.1444804e+03 1.82e-05 4.42e-03  -3.8 3.58e-03   0.1 1.00e+00 1.00e+00h  1
  17  4.1444799e+03 9.05e-05 3.06e-03  -3.8 7.43e-03  -0.4 1.00e+00 1.00e+00h  1
  18  4.1444616e+03 1.08e-05 5.76e-03  -5.7 3.72e-03  -0.9 1.00e+00 9.26e-01h  1
  19  4.1444608e+03 5.66e-06 1.11e-04  -5.7 2.43e-03  -1.3 1.00e+00 1.00e+00h  1
iter    objective    inf_pr   inf_du lg(mu)  ||d||  lg(rg) alpha_du alpha_pr  ls
  20  4.1444608e+03 2.21e-08 3.62e-05  -5.7 2.38e-03  -1.8 1.00e+00 1.00e+00H  1
  21  4.1444608e+03 3.35e-07 2.93e-05  -5.7 5.76e-03  -2.3 1.00e+00 1.00e+00H  1
  22  4.1444608e+03 2.25e-04 1.95e-05  -5.7 1.15e-02  -2.8 1.00e+00 1.00e+00h  1
  23  4.1444608e+03 5.41e-04 1.11e-05  -5.7 1.96e-02  -3.2 1.00e+00 1.00e+00h  1
  24  4.1444608e+03 2.03e-02 1.22e-04  -5.7 1.15e-01  -3.7 1.00e+00 1.00e+00h  1
  25  4.1444608e+03 8.07e-04 2.02e-05  -5.7 2.32e-02  -3.3 1.00e+00 1.00e+00h  1
  26  4.1444608e+03 6.02e-03 3.97e-05  -5.7 6.75e-02  -3.8 1.00e+00 1.00e+00h  1
  27  4.1444608e+03 1.93e-02 1.75e-04  -5.7 1.17e-01  -4.3 1.00e+00 1.00e+00h  1
  28  4.1444608e+03 4.88e-02 7.07e-05  -5.7 1.71e-01  -4.7 1.00e+00 1.00e+00h  1
  29  4.1444608e+03 8.42e-04 6.07e-06  -5.7 3.53e-02  -4.3 1.00e+00 1.00e+00h  1
iter    objective    inf_pr   inf_du lg(mu)  ||d||  lg(rg) alpha_du alpha_pr  ls
  30  4.1444608e+03 1.16e-02 4.10e-06  -5.7 1.11e-01  -4.8 1.00e+00 1.00e+00h  1
  31  4.1444608e+03 1.54e-03 1.71e-06  -5.7 3.87e-02  -4.4 1.00e+00 1.00e+00h  1
  32  4.1444608e+03 2.12e-02 9.93e-06  -5.7 1.38e-01  -4.8 1.00e+00 1.00e+00h  1
  33  4.1444608e+03 1.52e-01 8.57e-05  -5.7 5.25e-01  -5.3 1.00e+00 1.00e+00h  1
  34  4.1444608e+03 5.89e-03 2.06e-05  -5.7 1.27e-01  -4.9 1.00e+00 1.00e+00h  1
  35  4.1444608e+03 2.60e-01 1.06e-04  -5.7 1.23e+00    -  7.38e-01 1.00e+00H  1
  36  4.1444608e+03 4.78e-02 3.16e-05  -5.7 3.18e-01    -  1.00e+00 1.00e+00h  1
  37  4.1444608e+03 1.18e-03 3.22e-07  -5.7 7.50e-02    -  1.00e+00 1.00e+00h  1
  38  4.1444608e+03 3.36e-05 2.96e-09  -5.7 1.62e-02    -  1.00e+00 1.00e+00h  1
  39  4.1444608e+03 3.14e-10 5.68e-13  -5.7 5.87e-05    -  1.00e+00 1.00e+00h  1
iter    objective    inf_pr   inf_du lg(mu)  ||d||  lg(rg) alpha_du alpha_pr  ls
  40  4.1444605e+03 7.54e-08 1.83e-07  -7.0 2.81e-04    -  1.00e+00 1.00e+00h  1

Number of Iterations....: 40

                                   (scaled)                 (unscaled)
Objective...............:   9.2925124417578417e+01    4.1444605490239974e+03
Dual infeasibility......:   1.8347495281748061e-07    8.1829828956596351e-06
Constraint violation....:   1.3290786859965209e-08    1.3290786859965209e-08
Complementarity.........:   2.6885276874539710e-07    1.1990833486044711e-05
Overall NLP error.......:   2.6885276874539710e-07    1.1990833486044711e-05


Number of objective function evaluations             = 44
Number of objective gradient evaluations             = 41
Number of equality constraint evaluations            = 44
Number of inequality constraint evaluations          = 44
Number of equality constraint Jacobian evaluations   = 41
Number of inequality constraint Jacobian evaluations = 41
Number of Lagrangian Hessian evaluations             = 40
Total CPU secs in IPOPT (w/o function evaluations)   =      0.078
Total CPU secs in NLP function evaluations           =      0.026

EXIT: Optimal Solution Found.
=============================================================
Security-Constrained Optimal Power Flow
=============================================================
Number of contingencies             4
Multi-period contingencies?         NO
Solver                              IPOPT
Initialization                      MIDPOINT
Load loss allowed                   NO
Power imbalance allowed             NO
Ignore line flow constraints        NO


Convergence status                  CONVERGED
Objective value (base)              4144.46

----------------------------------------------------------------------
Bus        Pd      Qd      Vm      Va      mult_Pmis      mult_Qmis      Pslack         Qslack        
----------------------------------------------------------------------
1         0.00    0.00   1.100   0.000      2102.91         0.00         0.00         0.00
2         0.00    0.00   1.095   3.928      2059.18        -0.00         0.00         0.00
3         0.00    0.00   1.087   2.120      2065.15        -0.00         0.00         0.00
4         0.00    0.00   1.097  -1.993      2103.17         0.08         0.00         0.00
5        75.00   50.00   1.079  -3.060      2113.46         7.29         0.00         0.00
6        90.00   30.00   1.087  -3.927      2129.85         1.62         0.00         0.00
7         0.00    0.00   1.100   0.535      2059.57        -0.04         0.00         0.00
8       100.00   35.00   1.089  -1.720      2079.34         2.99         0.00         0.00
9         0.00    0.00   1.100  -0.135      2065.43        -0.09         0.00         0.00

----------------------------------------------------------------------------------------
From       To       Status     Sft      Stf     Slim     mult_Sf  mult_St 
----------------------------------------------------------------------------------------
1          4          1       73.18    72.98   380.00    -0.00    -0.00
2          7          1      114.18   114.68   250.00    -0.00    -0.00
3          9          1       83.58    84.61   300.00    -0.00    -0.00
4          5          1       29.69    40.50   250.00    -0.00    -0.00
4          6          1       44.86    46.03   250.00    -0.00    -0.00
5          7          1       51.29    49.04   250.00    -0.00    -0.00
6          9          1       48.94    51.43   150.00    -0.00    -0.00
7          8          1       66.61    68.11   250.00    -0.00    -0.00
8          9          1       38.87    34.15   150.00    -0.00    -0.00

----------------------------------------------------------------------------------------
Gen      Status     Fuel     Pg       Qg       Pmin     Pmax     Qmin     Qmax  
----------------------------------------------------------------------------------------
1          1    UNDEFINED    72.86     6.79    10.00   350.00  -300.00   300.00
2          1    UNDEFINED   114.07    -5.13    10.00   300.00  -300.00   300.00
3          1    UNDEFINED    80.21   -23.48    10.00   270.00  -300.00   300.00
[ExaGO] Finalizing scopflow application.
\end{lstlisting}

Example using HIOP solver with IPOPT subproblem solver.
\begin{lstlisting}
bin/scopflow -netfile $EXAGO_DIR/datafiles/case9/case9mod.m -ctgcfile $EXAGO_DIR/datafiles/case9/case9.cont -scopflow_Nc 4 -scopflow_solver IPOPT -print_output -scopflow_solver HIOP -scopflow_subproblem_solver IPOPT -scopflow_subproblem_model POWER_BALANCE_POLAR
[ExaGO] SCOPFLOW: Application created
[ExaGO] SCOPFLOW running with 5 subproblems (base case + 4 contingencies)
[ExaGO] SCOPFLOW: Using HIOP solver
Failed to read option file 'hiop_pridec.options'. Hiop will use default options.
[ExaGO] SCOPFLOW: Setup completed
total number of recourse problems  4

******************************************************************************
This program contains Ipopt, a library for large-scale nonlinear optimization.
 Ipopt is released as open source code under the Eclipse Public License (EPL).
         For more information visit http://projects.coin-or.org/Ipopt
******************************************************************************

This is Ipopt version 3.12.10, running with linear solver ma27.

Number of nonzeros in equality constraint Jacobian...:      114
Number of nonzeros in inequality constraint Jacobian.:       72
Number of nonzeros in Lagrangian Hessian.............:       96

Total number of variables............................:       24
                     variables with only lower bounds:        0
                variables with lower and upper bounds:       16
                     variables with only upper bounds:        0
Total number of equality constraints.................:       18
Total number of inequality constraints...............:       18
        inequality constraints with only lower bounds:        0
   inequality constraints with lower and upper bounds:       18
        inequality constraints with only upper bounds:        0

iter    objective    inf_pr   inf_du lg(mu)  ||d||  lg(rg) alpha_du alpha_pr  ls
   0  1.0318125e+04 1.80e+00 1.00e+02  -1.0 0.00e+00    -  0.00e+00 0.00e+00   0
   1  7.7157691e+03 1.17e+00 1.03e+02  -1.0 1.08e+00    -  6.27e-01 3.50e-01f  1
   2  7.6608235e+03 1.15e+00 1.01e+02  -1.0 6.28e+00    -  1.20e-02 1.37e-02f  1
   3  7.4466686e+03 1.09e+00 3.06e+02  -1.0 3.74e+00    -  4.15e-03 5.81e-02f  1
   4  5.4292675e+03 3.92e-01 4.83e+03  -1.0 7.34e-01    -  3.34e-03 6.40e-01f  1
   5  4.5792834e+03 2.24e-01 1.51e+03  -1.0 6.46e-01   2.0 8.77e-03 7.37e-01f  1
   6  4.2907579e+03 1.20e-02 3.57e+02  -1.0 3.36e-01    -  5.37e-01 1.00e+00f  1
   7  4.1690456e+03 4.40e-02 5.31e+01  -1.0 3.31e-01    -  9.22e-01 1.00e+00f  1
   8  4.1687926e+03 4.88e-04 1.93e+00  -1.0 4.79e-02   1.5 1.00e+00 1.00e+00h  1
   9  4.1497176e+03 1.19e-02 9.92e+00  -2.5 1.87e-01    -  8.38e-01 1.00e+00f  1
iter    objective    inf_pr   inf_du lg(mu)  ||d||  lg(rg) alpha_du alpha_pr  ls
  10  4.1463942e+03 1.09e-02 5.09e-01  -2.5 1.15e-01    -  8.71e-01 1.00e+00h  1
  11  4.1449657e+03 1.47e-03 1.79e-02  -2.5 2.75e-02    -  1.00e+00 1.00e+00h  1
  12  4.1445415e+03 6.63e-04 9.12e-02  -3.8 1.48e-02    -  1.00e+00 6.30e-01h  1
  13  4.1444705e+03 3.43e-04 4.96e-02  -3.8 2.08e-02    -  1.00e+00 8.93e-01h  1
  14  4.1444809e+03 4.48e-05 1.79e-04  -3.8 6.82e-03    -  1.00e+00 1.00e+00f  1
  15  4.1444611e+03 1.96e-05 4.55e-03  -5.7 4.57e-03    -  1.00e+00 9.34e-01h  1
  16  4.1444607e+03 6.49e-06 1.17e-05  -5.7 2.60e-03    -  1.00e+00 1.00e+00h  1
  17  4.1444605e+03 1.19e-06 2.07e-06  -7.0 1.11e-03    -  1.00e+00 1.00e+00h  1
  18  4.1444605e+03 1.58e-07 3.29e-07  -7.0 4.06e-04    -  1.00e+00 1.00e+00h  1

Number of Iterations....: 18

                                   (scaled)                 (unscaled)
Objective...............:   9.2925122354655841e+01    4.1444604570176507e+03
Dual infeasibility......:   3.2927387965389691e-07    1.4685615032563802e-05
Constraint violation....:   2.6639677713768961e-08    2.6639677713768961e-08
Complementarity.........:   4.7840038622596930e-07    2.1336657225678232e-05
Overall NLP error.......:   4.7840038622596930e-07    2.1336657225678232e-05


Number of objective function evaluations             = 19
Number of objective gradient evaluations             = 19
Number of equality constraint evaluations            = 19
Number of inequality constraint evaluations          = 19
Number of equality constraint Jacobian evaluations   = 19
Number of inequality constraint Jacobian evaluations = 19
Number of Lagrangian Hessian evaluations             = 18
Total CPU secs in IPOPT (w/o function evaluations)   =      0.027
Total CPU secs in NLP function evaluations           =      0.002

EXIT: Optimal Solution Found.
This is Ipopt version 3.12.10, running with linear solver ma27.

Number of nonzeros in equality constraint Jacobian...:      118
Number of nonzeros in inequality constraint Jacobian.:       64
Number of nonzeros in Lagrangian Hessian.............:      141

Total number of variables............................:       30
                     variables with only lower bounds:        0
                variables with lower and upper bounds:       22
                     variables with only upper bounds:        0
Total number of equality constraints.................:       24
Total number of inequality constraints...............:       16
        inequality constraints with only lower bounds:        0
   inequality constraints with lower and upper bounds:       16
        inequality constraints with only upper bounds:        0

iter    objective    inf_pr   inf_du lg(mu)  ||d||  lg(rg) alpha_du alpha_pr  ls
   0  0.0000000e+00 1.80e+00 0.00e+00  -1.0 0.00e+00    -  0.00e+00 0.00e+00   0
   1  0.0000000e+00 1.41e+00 3.55e+01  -1.0 9.96e-01    -  6.48e-01 2.16e-01h  1
   2  0.0000000e+00 1.36e+00 5.74e+01  -1.0 3.86e+00    -  3.79e-03 3.37e-02f  1
   3  0.0000000e+00 1.12e+00 3.23e+02  -1.0 1.01e+00    -  1.98e-02 1.76e-01f  1
   4  0.0000000e+00 4.03e-01 7.31e+02  -1.0 7.32e-01    -  2.71e-02 9.03e-01f  1
   5  0.0000000e+00 8.41e-02 1.58e+02  -1.0 6.95e-01    -  4.70e-01 8.52e-01h  1
   6  0.0000000e+00 4.93e-03 3.36e+01  -1.0 1.85e-01    -  1.00e+00 1.00e+00f  1
   7  0.0000000e+00 2.20e-02 2.50e+01  -1.0 6.50e-01    -  4.45e-01 2.50e-01h  3
   8  0.0000000e+00 1.07e-01 5.13e+00  -1.0 3.57e-01   0.0 1.00e+00 1.00e+00h  1
   9  0.0000000e+00 1.07e-01 4.86e+00  -1.7 9.45e-01    -  3.30e-01 1.98e-01h  2
iter    objective    inf_pr   inf_du lg(mu)  ||d||  lg(rg) alpha_du alpha_pr  ls
  10  0.0000000e+00 9.21e-03 3.47e+00  -1.7 1.17e-01  -0.5 8.84e-01 1.00e+00h  1
  11  0.0000000e+00 2.70e-02 9.88e-01  -1.7 1.67e-01    -  1.00e+00 1.00e+00h  1
  12  0.0000000e+00 8.46e-03 1.62e-01  -1.7 1.58e-01  -1.0 1.00e+00 1.00e+00h  1
  13  0.0000000e+00 7.11e-04 1.09e-01  -2.5 4.58e-02  -1.4 1.00e+00 1.00e+00h  1
  14  0.0000000e+00 1.11e-01 9.70e-02  -2.5 1.43e+00    -  1.00e+00 1.00e+00H  1
  15  0.0000000e+00 2.40e-02 5.99e-02  -2.5 2.27e-01  -1.9 1.00e+00 1.00e+00h  1
  16  0.0000000e+00 3.45e-02 2.41e-01  -2.5 3.69e-01    -  5.82e-01 1.00e+00h  1
  17  0.0000000e+00 1.85e-02 1.46e-01  -2.5 3.58e-01    -  1.00e+00 1.00e+00h  1
  18  0.0000000e+00 8.37e-03 2.78e-02  -2.5 9.85e-02    -  1.00e+00 1.00e+00h  1
  19  0.0000000e+00 6.33e-06 5.45e-04  -3.8 6.81e-03    -  1.00e+00 1.00e+00h  1
iter    objective    inf_pr   inf_du lg(mu)  ||d||  lg(rg) alpha_du alpha_pr  ls
  20  0.0000000e+00 1.42e-07 2.95e-06  -5.7 1.02e-03    -  1.00e+00 1.00e+00h  1
  21  0.0000000e+00 1.44e-06 1.25e-07  -7.0 3.43e-03    -  1.00e+00 1.00e+00h  1

Number of Iterations....: 21

                                   (scaled)                 (unscaled)
Objective...............:   0.0000000000000000e+00    0.0000000000000000e+00
Dual infeasibility......:   1.2466133995485293e-07    1.2466133995485293e-07
Constraint violation....:   1.8383034489088956e-07    1.8383034489088956e-07
Complementarity.........:   1.0156847334169061e-07    1.0156847334169061e-07
Overall NLP error.......:   1.8383034489088956e-07    1.8383034489088956e-07


Number of objective function evaluations             = 28
Number of objective gradient evaluations             = 22
Number of equality constraint evaluations            = 28
Number of inequality constraint evaluations          = 28
Number of equality constraint Jacobian evaluations   = 22
Number of inequality constraint Jacobian evaluations = 22
Number of Lagrangian Hessian evaluations             = 21
Total CPU secs in IPOPT (w/o function evaluations)   =      0.032
Total CPU secs in NLP function evaluations           =      0.003

EXIT: Optimal Solution Found.

...
...

=============================================================
Security-Constrained Optimal Power Flow
=============================================================
Number of contingencies             4
Multi-period contingencies?         NO
Solver                              HIOP
Initialization                      MIDPOINT
Load loss allowed                   NO
Power imbalance allowed             NO
Ignore line flow constraints        NO


Convergence status                  CONVERGED
Objective value (base)              4144.46

----------------------------------------------------------------------
Bus        Pd      Qd      Vm      Va      mult_Pmis      mult_Qmis      Pslack         Qslack        
----------------------------------------------------------------------
1         0.00    0.00   1.100   0.000      2102.91         0.00         0.00         0.00
2         0.00    0.00   1.095   3.928      2059.18        -0.00         0.00         0.00
3         0.00    0.00   1.087   2.120      2065.15        -0.00         0.00         0.00
4         0.00    0.00   1.097  -1.993      2103.16         0.08         0.00         0.00
5        75.00   50.00   1.079  -3.060      2113.45         7.29         0.00         0.00
6        90.00   30.00   1.087  -3.927      2129.85         1.62         0.00         0.00
7         0.00    0.00   1.100   0.535      2059.57        -0.04         0.00         0.00
8       100.00   35.00   1.089  -1.720      2079.34         2.99         0.00         0.00
9         0.00    0.00   1.100  -0.135      2065.43        -0.09         0.00         0.00

----------------------------------------------------------------------------------------
From       To       Status     Sft      Stf     Slim     mult_Sf  mult_St 
----------------------------------------------------------------------------------------
1          4          1       73.18    72.98   380.00    -0.00    -0.00
2          7          1      114.18   114.68   250.00    -0.00    -0.00
3          9          1       83.57    84.60   300.00    -0.00    -0.00
4          5          1       29.68    40.50   250.00    -0.00    -0.00
4          6          1       44.86    46.03   250.00    -0.00    -0.00
5          7          1       51.29    49.04   250.00    -0.00    -0.00
6          9          1       48.94    51.43   150.00    -0.00    -0.00
7          8          1       66.61    68.11   250.00    -0.00    -0.00
8          9          1       38.86    34.15   150.00    -0.00    -0.00

----------------------------------------------------------------------------------------
Gen      Status     Fuel     Pg       Qg       Pmin     Pmax     Qmin     Qmax  
----------------------------------------------------------------------------------------
1          1    UNDEFINED    72.86     6.79    10.00   350.00  -300.00   300.00
2          1    UNDEFINED   114.07    -5.13    10.00   300.00  -300.00   300.00
3          1    UNDEFINED    80.21   -23.47    10.00   270.00  -300.00   300.00
[ExaGO] Finalizing scopflow application.
\end{lstlisting}


\chapter{Stochastic optimal power flow (SOPFLOW)}\label{chap:sopflow}
SOPFLOW solves a stochastic security-constrained multi-period optimal power flow problem. The problem is set up as a two-stage optimization problem where the first-stage (base-case) represents the normal operation of the grid (or the most likely forecast) and the second-stage comprises of $N_s$ scenarios of forecast deviation. Each scenario can have multiple contingencies and each contingency can be multi-period. Thus, depending on the options selected, each stochastic scenario can be
\begin{itemize}
    \item Single-period, no contingency
    \item Single-period contingencies
    \item Multi-period contingencies
\end{itemize}

\section{Formulation}
An illustration of \sopflow in Fig. \ref{fig:sctopflow} for a case with two scenarios $s_0$ and $s_1$. Each scenario has two contingencies $c_0$, $c_1$, and each contingency has four time-periods.


\definecolor{lavander}{cmyk}{0,0.48,0,0}
\definecolor{violet}{cmyk}{0.79,0.88,0,0}
\definecolor{burntorange}{cmyk}{0,0.52,1,0}

\def\lav{lavander!90}
\def\oran{orange!30}

\tikzstyle{contingency}=[draw,circle,violet,bottom color=\lav,
                  top color= white, text=violet,minimum width=50pt]
\tikzstyle{base}=[draw,circle,burntorange, left color=\oran,
                       text=violet,minimum width=50pt]

\tikzstyle{time}=[draw,circle,blue,text=violet,minimum width=2pt]
\tikzstyle{tbase}=[draw,circle,burntorange, left color=\oran,
                            text=violet,minimum width=2pt]
                       
\tikzstyle{cedge}=[color=red]

\begin{figure}[h!]
\centering
\begin{tikzpicture}[auto, thick]
  % Place base case
  \node[base,label=above:$s_0c_0$] (s0c0) at (1,0) {};
  \node[time,label=below:$t_0$] (s0c0t0) at (0.6,0) {};
  \node[time,label=below:$t_1$] (s0c0t1) at (1.4,0) {};
  
  \node[contingency,label=right:$s_0c_1$] (s0c1) at (4,3) {};
  \node[time,label=below:$t_0$] (s0c1t0) at (3.6,3) {};
  \node[time,label=below:$t_1$] (s0c1t1) at (4.4,3) {};

  \node[contingency,label=right:$s_0c_2$] (s0c2) at (4,0) {};
  \node[time,label=below:$t_0$] (s0c2t0) at (3.6,0) {};
  \node[time,label=below:$t_1$] (s0c2t1) at (4.4,0) {};

  
  \node[contingency,label=right:$s_0c_3$] (s0c3) at (4,-3) {};
  \node[time,label=below:$t_0$] (s0c3t0) at (3.6,-3) {};
  \node[time,label=below:$t_1$] (s0c3t1) at (4.4,-3) {};
  
  \path (s0c0t0) edge (s0c0t1);
  \path (s0c1t0) edge (s0c1t1);
  \path (s0c2t0) edge (s0c2t1);
  \path (s0c3t0) edge (s0c3t1);
  
  \path[cedge] (s0c0) edge (s0c1);
  \path[cedge] (s0c0) edge (s0c2);
  \path[cedge] (s0c0) edge (s0c3);

  % Second scenario
  \node[base,label=above:$s_1c_0$] (s1c0) at (-2,0) {};
  \node[time,label=below:$t_0$] (s1c0t0) at (-2.4,0) {};
  \node[time,label=below:$t_1$] (s1c0t1) at (-1.6,0) {};
  
  \node[contingency,label=left:$s_1c_1$] (s1c1) at (-5,3) {};
  \node[time,label=below:$t_0$] (s1c1t0) at (-5.4,3) {};
  \node[time,label=below:$t_1$] (s1c1t1) at (-4.6,3) {};

  \node[contingency,label=left:$s_1c_2$] (s1c2) at (-5,0) {};
  \node[time,label=below:$t_0$] (s1c2t0) at (-5.4,0) {};
  \node[time,label=below:$t_1$] (s1c2t1) at (-4.6,0) {};

  
  \node[contingency,label=left:$s_1c_3$] (s1c3) at (-5,-3) {};
  \node[time,label=below:$t_0$] (s1c3t0) at (-5.4,-3) {};
  \node[time,label=below:$t_1$] (s1c3t1) at (-4.6,-3) {};
  
  \path (s1c0t0) edge (s1c0t1);
  \path (s1c1t0) edge (s1c1t1);
  \path (s1c2t0) edge (s1c2t1);
  \path (s1c3t0) edge (s1c3t1);
  
  \path[cedge] (s1c0) edge (s1c1);
  \path[cedge] (s1c0) edge (s1c2);
  \path[cedge] (s1c0) edge (s1c3);

  \path (s0c0) edge [ultra thick,color=blue] (s1c0);

\end{tikzpicture}

\caption{Stochastic multi-period contingency constrained structure with two scenarios $s_0$ and $s_1$. Each scenario has three contingencies $c_1$,$c_2$,and $c_3$. $s_0c_0$ and $s_1c_0$ denote the base-cases for the two scenarios. Each scenario and contingency has two time-periods $t_0$, and $t_2$, $t_2$. The {\textcolor{red}{red}} line denotes the coupling between the contingencies and their respective base-case scenarios.The {\textcolor{blue}{blue}} line denotes the coupling between the scenarios}
\label{fig:sctopflow}
\end{figure}


The formulation for the stochastic security-constrained multi-period optimal power flow is given in (\ref{eq:sctopflow_start}) -- (\ref{eq:sctopflow_end}). In this formulation, the objective is to reduce the expected cost, where $f(x_{s,0,0})$ is the cost for scenario $s$ with no contingencies (hence 0 for the contingency index). $\rho_s$ is the probability of scenario $s$.

\begin{align}
\centering
\text{min}&~\sum_{s=1}^{N_s-1}\pi_s\sum_{c=0}^{N_c-1}\sum_{t=0}^{N_t-1}f(x_{s,c,t})&  \label{eq:sctopflow_start}\\
&\text{s.t.}& \nonumber \\
&~g(x_{s,c,t}) = 0,                                        &s \in \left[1,N_s-1\right],c \in \left[0,N_c-1\right], t \in \left[0,N_t-1\right]& \\
&~h(x_{s,c,t}) \le 0,                                      &s \in \left[1,N_s-1\right],c \in \left[0,N_c-1\right], t \in \left[0,N_t-1\right]& \\
x^- & \le x_{s,c,t} \le x^+,                               &s \in \left[1,N_s-1\right],c \in \left[0,N_c-1\right], t\in \left[0,N_t-1\right]& \\
-\delta_t{x} & \le x_{s,c,t} - x_{s,c,t-\Delta{t}} \le \delta_t{x},&s \in \left[1,N_s-1\right],c \in \left[0,N_c-1\right], t \in \left[1,N_t-1\right]& \label{eq:sctopflow_time_coupling}\\
-\delta_c{x} & \le x_{s,c,0} - x_{s,0,0} \le \delta_c{x},&s \in \left[1,N_s-1\right],c \in \left[1,N_c-1\right]&
\label{eq:sctopflow_contingency_coupling} \\
-\delta_s{x} & \le x_{s,0,0} - x_{0,0,0} \le \delta_s{x},&s \in \left[1,N_s-1\right]&
\label{eq:sctopflow_end}
\end{align}

{\sopflow} uses all the modeling details used for modeling an optimal power flow problem, i.e., each of the circles shown in Fig. \ref{fig:sctopflow} has the modeling details of an optimal power flow problem (\opflow). Incorporating the probabilities $\pi_s$ for each scenario is not implemented yet which leads to each scenario having an equal probability. 

Currently, \sopflow uses wind power generation as the stochastic variables and each scenario is a realization of the power output from wind generators. A zero fuel cost is used for wind power generation to ensure wind generation would be the dispatched to the given target level (upper limit). 

For contingencies, \sopflow supports generation and/or transmission outages. A contingency can have multiple outages, but, it should not cause any islanding. The coupling between the no-contingency and the contingency case for each scenario is also the difference in real power output ($p_{jsct}^{\text{g}} - p_{js0t}^{\text{g}},~ \jinJgen$) that must be within the 30 minute generator ramp rate.

For multi time-period, we use ramping constraints on the generator real power output between successive time steps.

\sopflow can be run in two modes: preventive and corrective. In the preventive mode, generator real power output is fixed to the base-case values for generators at PV bus(es). In this mode, the generators at the reference bus provide/absorb any deficit/surplus power. The corrective mode allows deviation of the PV and PQ generator real power from the base-case dispatch constrained by its 30-min. ramp rate capability. Note that the preventive/corrective mode is only applied at the first step $t_0$. In the successive time-steps, the generator dispatch is dictated by the previous step dispatch and the ramp limits.

\section{Solvers}
\sopflow can be solved with \ipopt. If one wants to solve each scenario independently, i.e., without any coupling constraints then use \emph{EMPAR} solver. \emph{EMPAR} distributes the contingencies to different processes when executed in parallel.

\section{Input and Output}
The following files are needed for executing a SOPFLOW.
\begin{itemize}
    \item \textbf{Network file:} The network file describing the network details. Only \matpower format files are currently supported.
    \item \textbf{Scenario file:} \sopflow only supports reading wind generation scenarios in a CSV format. An example of this format for the 9-bus case is \href{https://gitlab.pnnl.gov/exasgd/frameworks/exago/-/tree/master/datafiles/case9/scenarios_9bus.csv}{here}.
    \item \textbf{Contingency file:} Contingencies can be specified via PTI format file as described in chapter \ref{chap:scopflow}.The option \lstinline{-sopflow_enable_multicontingency} should be set for multi-contingency problems.
    \item \textbf{Load data:} One file for load real power and one fo reactive power. The files need to be in CSV format. An example of the format for the 9-bus case is \href{https://gitlab.pnnl.gov/exasgd/frameworks/exago/-/tree/master/datafiles/case9}{here}.
\end{itemize}

The \sopflow output is saved to a directory named \emph{sopflowout}. This directory contains $N_s$ subdirectories to save the solution for each scenario. Each of these subdirectories contain $N_c$ subdirectories, one for each contingency. Each contingency subdirectory has $N_t$ MATPOWER format files to store the output for each time-period for the given contingency and scenario. The subdirectories have the directory name format \emph{scen\_x} where x is the scenario number,  \emph{cont\_y} where y is the contingency number, and the output files have the file name format \emph{t\_z} where z is the time-step number.

\section{Usage}
\begin{lstlisting}
    ./sopflow -netfile <netfilename>  -scenfile <scenfilename> \
    [-sopflow_enable_multicontingency 1] <sopflowoptions>
\end{lstlisting}

\section{Options}

\begin{table}[!htbp]
  \caption{SOPFLOW options}
  \small
  \begin{tabular}{|p{0.4\textwidth}|p{0.3\textwidth}|p{0.3\textwidth}|}
    \hline
    \textbf{Option} & \textbf{Meaning} & \textbf{Values (Default value)} \\ \hline
    -netfile & Network file name & string (\href{https://gitlab.pnnl.gov/exasgd/frameworks/exago/-/blob/master/datafiles/case9/case9mod_gen3_wind.m}{case9mod\_gen3\_wind.m}) \\ \hline
    -scenfile & Scenario file name & string (\href{https://gitlab.pnnl.gov/exasgd/frameworks/exago/-/blob/master/datafiles/case9/scenarios_9bus.csv}{case9/scenarios\_9bus.csv}) \\ \hline
    -sopflow\_mode & Operation mode: Preventive or corrective & 0 or 1 (0) \\ \hline
    -sopflow\_solver & Set solver for sopflow & IPOPT or  EMPAR \\ \hline
    -sopflow\_enable\_multicontingency & Multi-contingency SOPFLOW & 0 or 1 (0) \\ \hline \hline
    -ctgcfile & Contingency file name & string (\href{https://gitlab.pnnl.gov/exasgd/frameworks/exago/-/blob/master/datafiles/case9/case9.cont}{case9.cont}) \\ \hline
    -sopflow\_Ns & Number of scenarios & int (0. Passing -1 results in all scenarios in the file being used) \\ \hline
  \end{tabular}
  \label{tab:sopflow_options}
\end{table}

See table \ref{tab:sopflow_options}. With multi-contingency SOPFLOW, all \scopflow options given in Table \ref{tab:scopflow_options} can be used to tune the contingencies. All \opflow options in Table \ref{tab:opflow_options}, along with \tcopflow options in Table \ref{tab:tcopflow_options} can also be used. Multi-contingency SOPFLOW also allows the options listed after -sopflow\_enable\_multicontingency in Table \ref{tab:sopflow_options} to be used.

Depending on the \emph{mode}, SOPFLOW can either be \emph{preventive} (mode = 0) or \emph{corrective} (mode = 1). In the preventive mode, the PV and PQ generator real power is fixed to its corresponding base-case values. The generators at the reference bus pick up any make-up power required for the contingency. The corrective mode allows deviation of the PV and PQ generator real power from the base-case dispatch constrained by its 30-min. ramp rate capability.

\section{Examples}
Some \sopflow example runs are provided with some sample output. Options are the default options given in Tables \ref{tab:opflow_options}, \ref{tab:tcopflow_options}, \ref{tab:scopflow_options} and \ref{tab:sopflow_options} unless otherwise specified. Sample output is generated from running examples in the installation directory.

Example using the \ipopt solver:

\begin{lstlisting}
$ ./bin/sopflow -print_output -sopflow_solver IPOPT
[ExaGO INFO]: -- Checking ... -options_file not passed    exists: no
SOPFLOW: Application created
Rank[0]: color = 0, ns = 2, sstart= 0, send=2
SOPFLOW running with 2 scenarios (base case + 1 scenarios)
Rank 0 scenario range [0 -- 2]
SOPFLOW: Using IPOPT solver
SOPFLOW: Setup completed

********************************************************************
This program contains Ipopt, a library for large-scale nonlinear 
optimization. Ipopt is released as open source code under the 
Eclipse Public License (EPL).
********************************************************************

This is Ipopt version 3.12.10, running with linear solver ma27.

Number of nonzeros in equality constraint Jacobian...:      222
Number of nonzeros in inequality constraint Jacobian.:      168
Number of nonzeros in Lagrangian Hessian.............:      180

Total number of variables............................:       46
                     variables with only lower bounds:        0
                variables with lower and upper bounds:       30
                     variables with only upper bounds:        0
Total number of equality constraints.................:       37
Total number of inequality constraints...............:       50
        inequality constraints with only lower bounds:       12
   inequality constraints with lower and upper bounds:       38
        inequality constraints with only upper bounds:        0

iter  objective  inf_pr  inf_du lg(mu) ||d|| lg(rg) alpha_du alpha_pr  ls
0  1.4884250e+04 1.80e+00 5.02e+01  -1.0 0.00e+00 - 0.00e+00 0.00e+00   0
1  1.0889088e+04 1.45e+00 7.87e+01  -1.0 1.92e+00 - 1.69e-01 1.93e-01f  1
2  1.0548318e+04 1.33e+00 7.28e+01  -1.0 6.01e-01 - 2.76e-02 8.30e-02f  1
...
29  5.9075224e+03 1.24e-14 2.96e-12 -7.0 4.91e-10 - 1.00e+00 1.00e+00h  1

Number of Iterations....: 29

                                   (scaled)                 (unscaled)
Objective...........:   1.3245566021454471e+02    5.9075224455686939e+03
Dual infeasibility..:   2.9592247352450450e-12    1.3198142319192901e-10
Constraint violation:   1.2434497875801753e-14    1.2434497875801753e-14
Complementarity.....:   9.0910764708133484e-08    4.0546201059827535e-06
Overall NLP error...:   9.0910764708133484e-08    4.0546201059827535e-06


Number of objective function evaluations             = 33
Number of objective gradient evaluations             = 30
Number of equality constraint evaluations            = 33
Number of inequality constraint evaluations          = 33
Number of equality constraint Jacobian evaluations   = 30
Number of inequality constraint Jacobian evaluations = 30
Number of Lagrangian Hessian evaluations             = 29
Total CPU secs in IPOPT (w/o function evaluations)   =      0.006
Total CPU secs in NLP function evaluations           =      0.003

EXIT: Optimal Solution Found.
=============================================================
        Security-Constrained Optimal Power Flow
=============================================================
OPFLOW Formulation                  POWER_BALANCE_POLAR
Solver                              IPOPT
Initialization                      MIDPOINT
Number of scenarios                 1
Load loss allowed                   NO
Power imbalance allowed             NO
Ignore line flow constraints        NO

Number of variables                 48
Number of equality constraints      0
Number of inequality constraints    0
Number of coupling constraints      0

Convergence status                  CONVERGED
Objective value                     5907.52

----------------------------------------------------------------------
Bus        Pd      Qd      Vm      Va      mult_Pmis      mult_Qmis
----------------------------------------------------------------------
1         0.00    0.00   1.040   0.000      2218.76        -0.00
2         0.00    0.00   1.025   4.156      2170.20         0.00
3         0.00    0.00   1.025   1.685      2180.00         0.00
4         0.00    0.00   1.038  -2.389      2219.06        -0.56
5        75.00   30.00   1.026  -3.714      2231.93         2.96
6        90.00   30.00   1.023  -4.683      2252.92         4.40
7         0.00    0.00   1.033   0.291      2170.95         4.07
8       100.00   35.00   1.022  -2.352      2196.02         8.01
9         0.00    0.00   1.036  -0.689      2180.43         2.83

--------------------------------------------------------------------------
From       To       Status     Sft      Stf     Slim     mult_Sf  mult_St
--------------------------------------------------------------------------
1          4          1       78.31    78.15   380.00    -0.00    -0.00
2          7          1      114.58   115.48   250.00    -0.00    -0.00
3          9          1       76.90    77.69   300.00    -0.00    -0.00
4          5          1       30.36    36.13   250.00    -0.00    -0.00
4          6          1       47.82    49.77   250.00    -0.00    -0.00
5          7          1       45.94    49.28   250.00    -0.00    -0.00
6          9          1       45.11    47.63   150.00    -0.00    -0.00
7          8          1       68.77    69.83   250.00    -0.00    -0.00
8          9          1       37.83    31.76   150.00    -0.00    -0.00

-------------------------------------------------------------------------
Gen  Status    Fuel     Pg       Qg       Pmin     Pmax     Qmin     Qmax
-------------------------------------------------------------------------
1     1        COAL    78.13     5.42    10.00   350.00  -300.00   300.00
2     1        COAL   114.18    -9.55    10.00   300.00  -300.00   300.00
3     1        WIND    75.00   -17.00     0.00    75.00  -300.00   300.00
[ExaGO INFO]: Finalizing sopflow application.
\end{lstlisting}

Example using the \emph{EMPAR} solver with multicontingency enabled:

\begin{lstlisting}
$ ./bin/sopflow -sopflow_solver EMPAR -sopflow_enable_multicontingency 1
[ExaGO INFO]: -- Checking ... -options_file not passed       exists: no
SOPFLOW: Application created
Rank[0]: color = 0, ns = 2, sstart= 0, send=2
SOPFLOW running with 2 scenarios (base case + 1 scenarios)
Rank 0 scenario range [0 -- 2]
SOPFLOW: Using EMPAR solver
SCOPFLOW: Application created
SCOPFLOW running with 2 contingencies (base case + 1 contingencies)
Rank 0 has 2 contingencies, range [0 -- 2]
SCOPFLOW: Using IPOPT solver
SCOPFLOW: Setup completed
SCOPFLOW: Application created
SCOPFLOW running with 2 contingencies (base case + 1 contingencies)
Rank 0 has 2 contingencies, range [0 -- 2]
SCOPFLOW: Using IPOPT solver
SCOPFLOW: Setup completed
SOPFLOW: Setup completed

**********************************************************************
This program contains Ipopt, a library for large-scale nonlinear 
optimization. Ipopt is released as open source code under the 
Eclipse Public License (EPL).
**********************************************************************

This is Ipopt version 3.12.10, running with linear solver ma27.

Number of nonzeros in equality constraint Jacobian...:      226
Number of nonzeros in inequality constraint Jacobian.:      168
Number of nonzeros in Lagrangian Hessian.............:      215

Total number of variables............................:       51
                     variables with only lower bounds:        0
                variables with lower and upper bounds:       35
                     variables with only upper bounds:        0
Total number of equality constraints.................:       42
Total number of inequality constraints...............:       58
        inequality constraints with only lower bounds:       21
   inequality constraints with lower and upper bounds:       37
        inequality constraints with only upper bounds:        0

iter  objective  inf_pr  inf_du lg(mu) ||d|| lg(rg) alpha_du alpha_pr  ls
0  7.4421250e+03 1.80e+00 1.04e+01  -1.0 0.00e+00 - 0.00e+00 0.00e+00   0
1  6.5388630e+03 1.51e+00 2.24e+01  -1.0 1.07e+00 - 5.38e-01 1.58e-01f  1
2  5.9394843e+03 1.28e+00 2.56e+02  -1.0 7.32e-01 - 3.31e-02 1.56e-01f  1
...
29  3.0556401e+03 1.07e-14 5.62e-10  -8.6 4.33e-09 - 1.00e+00 1.00e+00f 1

Number of Iterations....: 29

                                   (scaled)                 (unscaled)
Objective...........:   6.8512110964678271e+01    3.0556401490246512e+03
Dual infeasibility..:   5.6231688979129259e-10    2.5079333284691651e-08
Constraint violation:   1.0658141036401503e-14    1.0658141036401503e-14
Complementarity.....:   2.7880113972390153e-09    1.2434530831686009e-07
Overall NLP error...:   2.7880113972390153e-09    1.2434530831686009e-07


Number of objective function evaluations             = 48
Number of objective gradient evaluations             = 30
Number of equality constraint evaluations            = 48
Number of inequality constraint evaluations          = 48
Number of equality constraint Jacobian evaluations   = 30
Number of inequality constraint Jacobian evaluations = 30
Number of Lagrangian Hessian evaluations             = 29
Total CPU secs in IPOPT (w/o function evaluations)   =      0.010
Total CPU secs in NLP function evaluations           =      0.006

EXIT: Optimal Solution Found.
This is Ipopt version 3.12.10, running with linear solver ma27.

Number of nonzeros in equality constraint Jacobian...:      226
Number of nonzeros in inequality constraint Jacobian.:      168
Number of nonzeros in Lagrangian Hessian.............:      215
Total number of variables............................:       51
                     variables with only lower bounds:        0
                variables with lower and upper bounds:       35
                     variables with only upper bounds:        0
Total number of equality constraints.................:       42
Total number of inequality constraints...............:       58
        inequality constraints with only lower bounds:       21
   inequality constraints with lower and upper bounds:       37
        inequality constraints with only upper bounds:        0

iter objective  inf_pr  inf_du lg(mu) ||d|| lg(rg) alpha_du alpha_pr  ls
0  7.4421250e+03 1.80e+00 1.04e+01  -1.0 0.00e+00 - 0.00e+00 0.00e+00   0
1  6.2852600e+03 1.53e+00 2.73e+01  -1.0 1.34e+00 - 3.21e-01 1.50e-01f  1
...
28  2.8488309e+03 1.07e-14 3.60e-09 -8.6 1.94e-08 - 1.00e+00 1.00e+00h  1

Number of Iterations....: 28

                                (scaled)                 (unscaled)
Objective...........:   6.3875131478884526e+01    2.8488308639582501e+03
Dual infeasibility..:   3.5985088591418374e-09    1.6049349511772596e-07
Constraint violation:   1.0658141036401503e-14    1.0658141036401503e-14
Complementarity.....:   4.7446211848960068e-09    2.1161010484636192e-07
Overall NLP error...:   4.7446211848960068e-09    2.1161010484636192e-07


Number of objective function evaluations             = 43
Number of objective gradient evaluations             = 29
Number of equality constraint evaluations            = 43
Number of inequality constraint evaluations          = 43
Number of equality constraint Jacobian evaluations   = 29
Number of inequality constraint Jacobian evaluations = 29
Number of Lagrangian Hessian evaluations             = 28
Total CPU secs in IPOPT (w/o function evaluations)   =      0.016
Total CPU secs in NLP function evaluations           =      0.009

EXIT: Optimal Solution Found.
\end{lstlisting}

Example using the \ipopt solver with multicontingency and multiperiod enabled:

\begin{lstlisting}
./bin/sopflow -sopflow_solver IPOPT -sopflow_enable_multicontingency 1\
-scopflow_enable_multiperiod 1 -sopflow_Ns -1 -scopflow_Nc -1
[ExaGO INFO]: -- Checking ... -options_file not passed        exists: no
SOPFLOW: Application created
Rank[0]: color = 0, ns = 4, sstart= 0, send=4
SOPFLOW running with 4 scenarios (base case + 3 scenarios)
Rank 0 scenario range [0 -- 4]
SOPFLOW: Using IPOPT solver
SCOPFLOW: Application created
SCOPFLOW running with 10 contingencies (base case + 9 contingencies)
Rank 0 has 10 contingencies, range [0 -- 10]
SCOPFLOW: Using IPOPT solver
TCOPFLOW: Application created
TCOPFLOW: Duration = 0.166667 hours, timestep = 5.000000 minutes, 
          number of time-steps = 3
TCOPFLOW: Using IPOPT solver
TCOPFLOW: Setup completed
TCOPFLOW: Application created
TCOPFLOW: Duration = 0.166667 hours, timestep = 5.000000 minutes, 
          number of time-steps = 3
TCOPFLOW: Using IPOPT solver
TCOPFLOW: Setup completed
...
SCOPFLOW running with 10 contingencies (base case + 9 contingencies)
Rank 0 has 10 contingencies, range [0 -- 10]
SCOPFLOW: Using IPOPT solver
...
TCOPFLOW: Setup completed
SCOPFLOW: Setup completed
SOPFLOW: Setup completed

**********************************************************************
This program contains Ipopt, a library for large-scale nonlinear 
optimization. Ipopt is released as open source code under the 
Eclipse Public License (EPL).
**********************************************************************

This is Ipopt version 3.12.10, running with linear solver ma27.

Number of nonzeros in equality constraint Jacobian...:    12678
Number of nonzeros in inequality constraint Jacobian.:     9636
Number of nonzeros in Lagrangian Hessian.............:    10404

Total number of variables............................:     2688
                     variables with only lower bounds:        0
                variables with lower and upper bounds:     1728
                     variables with only upper bounds:        0
Total number of equality constraints.................:     2223
Total number of inequality constraints...............:     2922
        inequality constraints with only lower bounds:      648
   inequality constraints with lower and upper bounds:     2274
        inequality constraints with only upper bounds:        0

iter  objective  inf_pr  inf_du lg(mu) ||d|| lg(rg) alpha_du alpha_pr  ls
0  8.0374950e+05 2.59e+00 6.64e+01  -1.0 0.00e+00 - 0.00e+00 0.00e+00   0
1  7.8996010e+05 2.48e+00 6.98e+01 -1.0 1.43e+00 2.0 2.04e-02 3.97e-02f 1
2  7.7076082e+05 2.31e+00 9.50e+01 -1.0 1.16e+00 2.4 2.69e-02 6.75e-02f 1
...
197 4.2506331e+05 1.42e-14 3.34e-10 -7.0 2.10e-09 - 1.00e+00 1.00e+00h  1

Number of Iterations....: 197

                                (scaled)                 (unscaled)
Objective...........:   9.5305673828307717e+03    4.2506330527425243e+05
Dual infeasibility..:   3.3401254461705077e-10    1.4896959489920465e-08
Constraint violation:   1.4210854715202004e-14    1.4210854715202004e-14
Complementarity.....:   9.1353878464689626e-08    4.0743829795251576e-06
Overall NLP error...:   9.1353878464689626e-08    4.0743829795251576e-06


Number of objective function evaluations             = 283
Number of objective gradient evaluations             = 198
Number of equality constraint evaluations            = 283
Number of inequality constraint evaluations          = 283
Number of equality constraint Jacobian evaluations   = 198
Number of inequality constraint Jacobian evaluations = 198
Number of Lagrangian Hessian evaluations             = 197
Total CPU secs in IPOPT (w/o function evaluations)   =      2.136
Total CPU secs in NLP function evaluations           =      4.421

EXIT: Optimal Solution Found.
[ExaGO INFO]: Finalizing sopflow application.
\end{lstlisting}



\chapter{Power flow (\pflow)}\label{chap:pflow}
\todo
\section{Input}
\section{Formulation}
\section{Output}
\section{Usage}
\section{Options}
\section{Example}


\begin{appendices}

\newcommand{\notationtable}[4] {
  {
    \tabulinesep=3pt
    \begin{center}
      \begin{longtabu} to \linewidth {lX[L]}

        % first page header
        \caption[
          %short caption for list of tables
        ]{
          #3 %long caption
          #4 %label
        } \\
        \toprule
        #2 %header
        \midrule
        \endfirsthead

        % further pages header
        \caption{Continued} \\
        \toprule
        #2 %header
        \midrule
        \endhead

        % first pages footer
        \bottomrule
        \endfoot

        % last page footer
        \bottomrule
        \endlastfoot
        
        #1 % main table content
      \end{longtabu}
    \end{center}
  }
}

\chapter{Symbol reference}\label{chap:symbref}

Units of measurement are given in Table (\ref{tbl:units}),
indices and index sets in Table (\ref{tbl:index_sets}),
subsets in Table (\ref{tbl:subsets}),
special set elements in Table (\ref{tbl:elements}),
real-valued parameters in Table (\ref{tbl:params}),
functions in Table (\ref{tbl:fcns}),
and optimization model variables in Table (\ref{tbl:vars}).

\notationtable{
  1 & dimensionless. Dimensionless real number quantities are indicated by a unit of 1. \\
  %bin & binary. Binary quantities, i.e. taking values in $\{0,1\}$, are indicated by a unit of bin. \\
  USD & US dollar. Cost, penalty, and objective values are expressed in USD. \\
  h & hour. Time is expressed in h. \\
  pu & per unit. Voltage magnitude is expressed in a per unit system under given
  base values, and the unit is denoted by pu \\
  %kV & kilovolt. In the physical unit convention,
  %voltage magnitude is expressed in kV. \\
  rad & radian. Voltage angles are expressed in rad. \\
  MW & megawatt. Real power is expressed in MW. \\
  MVar & megavolt-ampere-reactive. Reactive power is expressed in MVar. \\
  MVA & megavolt-ampere. Apparent power is expressed in MVA. \\
  MW at 1 pu & megawatt at unit voltage. Conductance is expressed in MW at 1 pu, meaning the conductance is such as to yield a real power flow equal to the indicated amount when the voltage is equal to 1 pu \\
  MVar at 1 pu & megavolt-ampere-reactive at unit voltage. Susceptance is expressed in MVar at 1 pu, meaning the susceptance is such as to yield a reactive power flow equal to the indicated amount when the voltage is equal to 1 pu \\
} {
  Symbol & Description \\
} {
  Units of measurement
} {
  \label{tbl:units}
}

\notationtable{
  $a \in A$ & areas \\
  $i \in I$ & buses \\
  $j \in J$ & bus-connected grid components, i.e. loads, shunts, generators, stochastic resources, branches \\
  $k \in K$ & security contingencies, i.e. NERC $(n-1)$- or $(n-k)$-style contingencies,
  different from the severe event we are modeling \\
  %$n \in N$ & nodes in the multistage stochastic scenario tree induced by $S^{\text{equiv}}_{ts}$, $t \in T$, $s \in S$ \\
  $s \in S$ & stochastic scenarios \\
  $t \in T$ & time periods \\
} {
  Symbol & Description \\
} {
  Index sets
} {
  \label{tbl:index_sets}
}

\notationtable{
  $I_{ts} \subset I$ & buses in the main connected component in time period $t$,
  scenario $s$ \\
  $I_{tsk} \subset I$ & buses in the main connected component in time period $t$,
  scenario $s$, contingency $k$ \\
  $J^{\text{gen}} \subset J$ & generators \\
  $J^{\text{ld}} \subset J$ & loads \\
  $J^{\text{fxld}} \subset J^{\text{ld}}$ & fixed loads \\
  $J^{\text{cld}} \subset J^{\text{ld}}$ & curtailable loads \\
  $J^{\text{dld}} \subset J^{\text{ld}}$ & delayable loads \\
  $J^{\text{frld}} \subset J^{\text{ld}}$ & frequency-responsive loads \\
  $J^{\text{br}} \subset J$ & branches, i.e. lines, transformers \\
  $J^{\text{sh}} \subset J$ & shunts \\
  $J^{\text{sr}} \subset J$ & stochastic resources \\
  $J_i \subset J$ & 1-bus components (i.e. all except branches) connected to bus $i$ \\
  $J^{\text{o}}_i \subset J$ & branches with origin bus at bus $i$ \\
  $J^{\text{d}}_i \subset J$ & branches with destination bus at bus $i$ \\
  $J_{ts} \subset J$ & components in service pre-contingency
  in period $t$, scenario $s$ \\
  $J_{tsk} \subset J$ & components in service post-contingency
  in period $t$, scenario $s$,
  contingency $k$ \\
  $J^{\text{gen}}_{ts} \subset J$ & generators in service pre-contingency
  in period $t$, scenario $s$ \\
  $J^{\text{ld}}_{ts} \subset J$ & loads in service pre-contingency
  in period $t$, scenario $s$ \\
  $J^{\text{br}}_{ts} \subset J$ & branches in service pre-contingency
  in period $t$, scenario $s$ \\
  $J^{\text{sh}}_{ts} \subset J$ & shunts in service pre-contingency
  in period $t$, scenario $s$ \\
  $J^{\text{sr}}_{ts} \subset J$ & stochastic resources in service pre-contingency
  in period $t$, scenario $s$ \\
  $J^{\text{gen}}_{tsk} \subset J$ & generators in service post-contingency
  in period $t$, scenario $s$, contingency $k$ \\
  $J^{\text{ld}}_{tsk} \subset J$ & loads in service post-contingency
  in period $t$, scenario $s$, contingency $k$ \\
  $J^{\text{br}}_{tsk} \subset J$ & branches in service post-contingency
  in period $t$, scenario $s$, contingency $k$ \\
  $J^{\text{sh}}_{tsk} \subset J$ & shunts in service post-contingency
  in period $t$, scenario $s$, contingency $k$ \\
  $J^{\text{sr}}_{tsk} \subset J$ & stochastic resources in service post-contingency
  in period $t$, scenario $s$, contingency $k$ \\
  $J^{\text{gen}}_{its} \subset J$ & generators in service pre-contingency
  in period $t$, scenario $s$, and connected to bus $i$ \\
  $J^{\text{ld}}_{its} \subset J$ & loads in service pre-contingency
  in period $t$, scenario $s$, and connected to bus $i$ \\
  $J^{\text{o}}_{its} \subset J$ & branches in service pre-contingency
  in period $t$, scenario $s$, with origin bus at bus $i$ \\
  $J^{\text{d}}_{its} \subset J$ & branches in service pre-contingency
  in period $t$, scenario $s$, with destination bus at bus $i$ \\
  $J^{\text{sh}}_{its} \subset J$ & shunts in service pre-contingency
  in period $t$, scenario $s$, and connected to bus $i$ \\
  $J^{\text{sr}}_{its} \subset J$ & stochastic resources in service pre-contingency
  in period $t$, scenario $s$, and connected to bus $i$ \\
  $J^{\text{gen}}_{itsk} \subset J$ & generators in service post-contingency
  in period $t$, scenario $s$, contingency $k$, and connected to bus $i$ \\
  $J^{\text{ld}}_{itsk} \subset J$ & loads in service post-contingency
  in period $t$, scenario $s$, contingency $k$, and connected to bus $i$ \\
  $J^{\text{o}}_{itsk} \subset J$ & branches in service post-contingency
  in period $t$, scenario $s$, contingency $k$, with origin bus at bus $i$ \\
  $J^{\text{d}}_{itsk} \subset J$ & branches in service post-contingency
  in period $t$, scenario $s$, contingency $k$, with destination bus at bus $i$ \\
  $J^{\text{sh}}_{itsk} \subset J$ & shunts in service post-contingency
  in period $t$, scenario $s$, contingency $k$, and connected to bus $i$ \\
  $J^{\text{sr}}_{itsk} \subset J$ & stochastic resources in service post-contingency
  in period $t$, scenario $s$, contingency $k$, and connected to bus $i$ \\
  $K_{ts} \subset K$ & contingencies that are enforced in each time period $t$
  and scenario $s$ \\
  $S^{\text{equiv}}_{ts} \subset S$ & stochastic scenarios (including $s$ itself)
  that are indistinguishable from scenario $s$ through period $t$ \\
} {
  Symbol & Description \\
} {
  Subsets
} {
  \label{tbl:subsets}
}

\notationtable{
  $a_i \in A$ & area of bus $i$ \\
  $a_j \in A$ & area of 1-bus component $j$ \\
  $i_j \in I$ & connection bus of 1-bus component $j \in J \setminus J^{\text{br}}$ \\
  $i^{\text{d}}_j \in I$ & destination bus of branch $j \in J^{\text{}}$ \\
  $i^{\text{o}}_j \in I$ & origin bus of branch $j \in J^{\text{br}}$ \\
} {
  Symbol & Description \\
} {
  Special set elements
} {
  \label{tbl:elements}
}

\notationtable{
  $b^{\text{max}}_j$ & maximum susceptance for shunt
  $j \in J^{\text{sh}}$ (MVar at 1 pu) \\
  $b^{\text{min}}_j$ & minimum susceptance for shunt
  $j \in J^{\text{sh}}$ (MVar at 1 pu) \\
  $b^{\text{ch}}_j$ & charging susceptance for branch
  $j \in J^{\text{br}}$ (MVar at 1 pu) \\
  $b^{\text{s}}_j$ & series susceptance for branch
  $j \in J^{\text{br}}$ (MVar at 1 pu) \\
  $c^{\text{fr}}$ & marginal cost of frequency deviation (USD/hz/h) \\
  $c^{\text{fr}}_a$ & marginal cost of frequency deviation in area $a$ (USD/hz/h) \\
  $c^{\text{gen}}_j$ & marginal cost of energy generation for generator $j$ (USD/MW/h) \\
  $c^{\text{curt}}_j$ & marginal cost of load curtailment for curtailable load $j$ (USD/MW/h) \\
  $c^{\text{del}}_j$ & marginal cost of load delay for delayable load $j$ (USD/MW/h) \\ % todo: further 1/h dimension?
  $c^{\text{shed}}_j$ & marginal cost of load shedding for load $j$ (USD/MW/h) \\ %???? todo remove, and other references to coordinated load shedding
  $c^{\text{s,viol}}_j$ & cost of flow limit violation for branch $j \in J^{\text{br}}$ (USD/MVA/h) \\
  $d_t$ & duration of time period $t$ (h) \\
  %$d^{\text{lead}}_j$ & planning lead time of generator $j$ (h) \\
  $g^{\text{max}}_j$ & maximum conductance for shunt
  $j \in J^{\text{sh}}$ (MW at 1 pu) \\
  $g^{\text{min}}_j$ & minimum conductance for shunt
  $j \in J^{\text{sh}}$ (MW at 1 pu) \\
  $g^{\text{s}}_j$ & series conductance for branch $j \in J^{\text{br}}$
  (MW at 1 pu) \\
  $p^{\text{set,0}}_j$ & real power set point of generator $j \in J^{\text{gen}}$ in prior time period (MW) \\
  $p^{\text{max}}_j$ & maximum real power output for generator
  $j \in J^{\text{gen}}$ (MW) \\
  $p^{\text{min}}_j$ & minimum real power output for generator
  $j \in J^{\text{gen}}$ (MW) \\
  $p^{\text{r}}_j$ & maximum ramp rate for generator $j \in J^{\text{gen}}$ (MW/h) \\
  $p^{\text{sm}}$ & smoothing parameter for generator frequency response (MW) \\
  $p^{\text{ld},0}_{jts}$ & nominal real power consumption of load $j$ in time period $t$,
  scenario $s$ (MW) \\
  $q^{\text{max}}_j$ & maximum reactive power output for generator
  $j \in J^{\text{gen}}$ (MVar)\\
  $q^{\text{min}}_j$ & minimum reactive power output for generator
  $j \in J^{\text{gen}}$ (MVar) \\
  $q^{\text{ld},0}_{jts}$ & nominal reactive power consumption of load $j$ in time period $t$,
  scenario $s$ (MW) \\
  $s^{\text{max}}_j$ & maximum apparent power flow for branch $j \in J^{\text{br}}$
  (MVA) \\
  $s^{\text{max,sc}}_j$ & maximum apparent power flow for branch $j \in J^{\text{br}}$
  in security contingencies (MVA) \\
  $v^{\text{max}}_i$ & maximum voltage magnitude for bus $i \in I$ (pu) \\
  $v^{\text{min}}_i$ & minimum voltage magnitude for bus $i \in I$ (pu) \\
  $v^{\text{max,sc}}_i$ & maximum voltage magnitude for bus $i \in I$
  in security contingencies (pu) \\
  $v^{\text{min,sc}}_i$ & minimum voltage magnitude for bus $i \in I$
  in security contingencies (pu) \\
  $\alpha^{\text{max}}_j$ & upper bound on $\alpha_j$ (1) \\
  $\alpha^{\text{min}}_j$ & lower bound on $\alpha_j$ (1) \\
  $\alpha^{\text{sc,max}}_j$ & upper bound on $\alpha^{\text{sc}}_j$ (1) \\
  $\alpha^{\text{sc,min}}_j$ & lower bound on $\alpha^{\text{sc}}_j$ (1) \\
  $\beta^{\text{max}}_j$ & upper bound on $\beta_j$ (1) \\
  $\beta^{\text{min}}_j$ & lower bound on $\beta_j$ (1) \\
  $\alpha_j$ & proportional frequency response coefficient of generator
  $j \in J^{\text{gen}}$ or load $j \in J^{\text{ld}}$ (MW/hz) \\
  $\alpha^{\text{sc}}_j$ & coefficient for post-contingency proportional frequency
  response (i.e. droop coefficient) of generator
  $j \in J^{\text{gen}}$ (MW/hz) \\
  $\beta^{\text{ld}}_j$ & proportional frequency response coefficient of
  load $j \in J^{\text{ld}}$ (1/hz) \\
  $\delta^{\text{sm}}$ & smoothing parameter for load shedding frequency response (hz) \\
  $\delta^{\text{tol}}$ & load shedding frequency deviation tolerance \\
  $\gamma$ & general smoothing parameter (1) \\
  %$f^{\text{nom}}$ & system nominal frequency (hz) \\
  $\phi^{\text{max}}_j$ & maximum phase shift for branch
  $j \in J^{\text{br}}$ (rad) \\
  $\phi^{\text{min}}_j$ & minimum phase shift for branch
  $j \in J^{\text{br}}$ (rad) \\
  $\pi_s$ & probability of scenario $s$ (1) \\
  $\tau^{\text{max}}_j$ & maximum tap ratio for branch $j \in J^{\text{br}}$ (1) \\
  $\tau^{\text{min}}_j$ & minimum tap ratio for branch $j \in J^{\text{br}}$ (1) \\
} {
  Symbol & Description \\
} {
  Real-valued parameters
} {
  \label{tbl:params}
}

\notationtable{
  $\psi(x_1, x_2)$ & a smooth approximation of $\max(x_1, x_2)$,
  e.g. $\psi(x_1, x_2) = (x_1 + x_2 + (1 + (x_1 - x_2)^2)^{1/2})/2$
  (input and output both have dimensions of 1, i.e. dimensionless) \\
  $\psi_{\gamma}(x_1, x_2)$ & dilation of $\psi$ by smoothing parameter $\gamma > 0$
  i.e. $\psi_{\gamma}(x_1, x_2) = \gamma \psi(x_1/\gamma, x_2/\gamma)$
} {
  Symbol & Description \\
} {
  Functions
} {
  \label{tbl:fcns}
}

\notationtable{
  $b_{jts}$ & susceptance of shunt $j \in J^{\text{sh}}$ in time period $t$,
  scenario $s$ (MVar at 1 pu) \\
  $g_{jts}$ & conductance of shunt $j \in J^{\text{sh}}$ in time period $t$,
  scenario $s$ (MW at 1 pu) \\
  $p^{\text{d}}_{jts}$ & pre-contingency real power flow at the destination bus into
  branch $j \in J^{\text{br}}$ in period $t$,
  scenario $s$ (MW) \\
  $p^{\text{d}}_{jtsk}$ & post-contingency real power flow at the destination bus into
  branch $j \in J^{\text{br}}$ in period $t$,
  scenario $s$, contingency $k$ (MW) \\
  $p^{\text{gen}}_{jts}$ & pre-contingency real power output (reflecting frequency deviation)
  of generator $j \in J^{\text{gen}}$ in time period $t$, scenario $s$ (MW) \\
  $p^{\text{gen}}_{jtsk}$ & post-contingency real power output (reflecting frequency deviation
  and droop control)
  of generator $j \in J^{\text{gen}}$ in time period $t$, scenario $s$,
  contingency $k$ (MW) \\
  $p^{\text{ld}}_{jts}$ & real power consumption (reflecting pre-contingency
  frequency deviation) of
  load $j \in J^{\text{ld}}$ in time period $t$, scenario $s$ \\
  $p^{\text{curt}}_{jts}$ & real power curtailment of curtailable
  load $j \in J^{\text{cld}}$ in time period $t$, scenario $s$ \\
  $p^{\text{del}}_{jts}$ & real power delay of delayable
  load $j \in J^{\text{dld}}$ in time period $t$, scenario $s$ \\
  $p^{\text{o}}_{jts}$ & pre-contingency real power flow at the origin bus into
  branch $j \in J^{\text{br}}$ in period $t$,
  scenario $s$ (MW) \\
  $p^{\text{o}}_{jtsk}$ & post-contingency real power flow at the origin bus into
  branch $j \in J^{\text{br}}$ in period $t$,
  scenario $s$, contingency $k$ (MW) \\
  %$p^{\text{set}}_{jt}$ & nonanticipative real power output setpoint (value not reflecting frequency deviation)
  %of generator $j \in J^{\text{gen}}$ in period $t$, during generator dispatch lead time (MW) \\
  $p^{\text{set}}_{jts}$ & real power output setpoint (value not reflecting frequency deviation)
  of generator $j \in J^{\text{gen}}$ in period $t$, scenario $s$ (MW) \\
  $p^{\text{sr}}_{jts}$ & real power output of stochastic resource $j$
  in period $t$, scenario $s$ (MW) \\
  $q^{\text{d}}_{jts}$ & pre-contingency reactive power flow at the destination bus into
  branch $j \in J^{\text{br}}$ in period $t$,
  scenario $s$ (MVar) \\
  $q^{\text{d}}_{jtsk}$ & post-contingency reactive power flow at the destination bus into
  branch $j \in J^{\text{br}}$ in period $t$,
  scenario $s$, contingency $k$ (MVar) \\
  $q^{\text{gen}}_{jts}$ & pre-contingency reactive power output
  of generator $j \in J^{\text{gen}}$ in time period $t$, scenario $s$ \\
  $q^{\text{gen}}_{jtsk}$ & post-contingency reactive power output
  of generator $j \in J^{\text{gen}}$ in time period $t$, scenario $s$,
  contingency $k$ (MVar) \\
  $q^{\text{ld}}_{jts}$ & reactive power consumption (reflecting pre-contingency
  frequency deviation) of
  load $j \in J^{\text{ld}}$ in time period $t$, scenario $s$ (MVar) \\
  $q^{\text{o}}_{jts}$ & pre-contingency reactive power flow at the origin bus into
  branch $j \in J^{\text{br}}$ in period $t$,
  scenario $s$ (MVar) \\
  $q^{\text{o}}_{jtsk}$ & post-contingency reactive power flow at the origin bus into
  branch $j \in J^{\text{br}}$ in period $t$,
  scenario $s$, contingency $k$ (MVar) \\
  $r^{\text{\text{shed}}}_{jts}$ & fraction
  of the nominal value of load $j$ that is shed in time period $t$ and scenario $s$ (1) \\
  $s^{\text{viol}}_{jts}$ & pre-contingency
  violation of flow limit of branch $j \in j^{\text{br}}$
  in time period $t$, scenario $s$, (MVA) \\
  $s^{\text{viol}}_{jtsk}$ & post-contingency
  violation of flow limit of branch $j \in j^{\text{br}}$
  in time period $t$, scenario $s$,
  contingency $k$ (MVA) \\
  $v_{its}$ & pre-contingency voltage magnitude of bus $i$ in period $t$,
  scenario $s$ (pu) \\
  $v_{itsk}$ & post-contingency voltage magnitude of bus $i$ in period $t$,
  scenario $s$, contingency $k$ (pu) \\
  $w_{ts}$ & stochastic objective rate in period $t$, scenario $s$ (USD/h) \\
  $w^{\text{fr}}_{ts}$ & stochastic objective rate, frequency term,
  in period $t$, scenario $s$ (USD/h) \\
  $w^{\text{\text{shed}}}_{ts}$ & stochastic obective rate, load shedding term, in period $t$, scenario $s$ (USD/h) \\
  $w^{\text{viol}}_{ts}$ & stochastic objective rate, pre-contingency line and
  transformer flow limit violation term,
  in period $t$, scenario $s$ (USD/h) \\
  $w^{\text{viol}}_{tsk}$ & stochastic objective rate, post-contingency line and
  transformer flow limit violation term,
  in period $t$, scenario $s$, contingency $k$ (USD/h) \\
  $z$ & total minimization objective (expected value, time-integrated) (USD) \\
  $z_s$ & stochastic objective in scenario $s$ (USD) \\
  $\delta_{ats}$ & pre-contingency local frequency deviation
  (i.e. nominal frequency minus actual frequency)
  in area $a$, period $t$, scenario $s$ (hz) \\
  $\delta_{atsk}$ & post-contingency local frequency deviation
  (i.e. nominal minus actual)
  in area $a$, period $t$, scenario $s$, contingency $k$ (hz) \\
  $\delta_{ts}$ & pre-contingency system frequency deviation
  (i.e. nominal frequency minus actual frequency)
  in period $t$, scenario $s$ (hz) \\
  $\delta_{tsk}$ & post-contingency system frequency deviation
  (i.e. nominal minus actual)
  in period $t$, scenario $s$, contingency $k$ (hz) \\
  $\phi_{jts}$ & phase difference of branch $j \in J^{\text{br}}$ in time $t$,
  scenario $s$ (rad) \\
  $\tau_{jts}$ & tap ratio of branch $j \in J^{\text{br}}$ in time $t$,
  scenario $s$ (1) \\
  $\theta_{its}$ & pre-contingency
  voltage angle of bus $i$ in period $t$, scenario $s$ (rad) \\
  $\theta_{itsk}$ & post-contingency voltage angle of bus $i$ in period $t$, scenario $s$,
  contingency $k$(rad) \\
} {
  Symbol & Description \\
} {
  Optimization model variables
} {
  \label{tbl:vars}
}



\end{appendices}

\bibliographystyle{plain}
\bibliography{manual-bib}

\end{document}
