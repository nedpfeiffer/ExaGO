\section{System requirements}

\exago is currently only built on 64b OSX and Linux machines, compiled with GCC $>= 7.3$.
We build \exago on Intel and IBM Power9 architectures.

\section{Prerequisites}

This section assumes that you already have the \exago source code, and that the environment variable \texttt{EXAGODIR} is the directory of the \exago source code.
\exago may be acquired via \href{https://gitlab.pnnl.gov/exasgd/frameworks/exago}{the PNNL git repository linked here}, like so:

\begin{lstlisting}[language=bash]
> git clone https://gitlab.pnnl.gov/exasgd/frameworks/exago.git exago
> export EXAGODIR=$PWD/exago
\end{lstlisting}

Paths to installations of third party software in examples are abbreviated with placeholder paths.
For example, \texttt{/path/to/cuda} is a placeholder for a path to a valid \texttt{CUDA Toolkit} installation.

\section{Dependencies}

\exago has dependencies in table \ref{tab:deps}.

\begin{table}[h]
  \caption{\label{tab:deps}Dependency Table}
  \begin{tabular}{|l|c|c|l|}
    \hline
    \textbf{Dependency} & \textbf{Version Constraints} & \textbf{Mandatory} & \textbf{Notes} \\
    \hline
    \petsc & $>= 3.13.0$ & \checkmark & Only needed for the setup stage \\ \hline
    \cmake & $>= 3.10$ & \checkmark & Only a build dependency \\ \hline
    MPI & $>= 3.1.3$ & & Only tested with \texttt{openmpi} and \texttt{spectrummpi} \\ \hline
    \ipopt & $>= 3.12$ & & \\ \hline
    \hiop & $>= 0.3.0$ & & Prefer dynamically linked \\ \hline
    \raja & $>= 0.11.0$ & & \\ \hline
    \href{https://github.com/LLNL/umpire}{Umpire \cite{umpire}} & $>= 2.1.0$ & & Only when RAJA is enabled \\ \hline
    \magma & $>= 2.5.2$ & & Only when GPU acceleration is enabled \\ \hline
    \cuda & $>= 10.2.89$ & & Only when GPU acceleration is enabled \\
    \hline
  \end{tabular}
  
\end{table}

These may all be toggled via \cmake which will be discussed in the section \hyperref[sec:building_and_installation]{Building and installation}.

\subsection{Notes on environment modules}

Many of the dependencies are available via environment modules on institutional clusters.
To get additional information on your institution's clusters, please ask your institution's system administrators.
Some end-to-end examples in this document will use system-specific modules and are not expected to expected to run on other clusters.

For example, the modules needed to build and run \exago on Newell, an IBM Power9 PNNL cluster, are as follows:

\begin{lstlisting}[language=bash]
> module load gcc/7.4.0
> module load openmpi/3.1.5
> module load cuda/10.2
> module load magma/2.5.2_cuda10.2
> module load metis/5.1.0
> module load cmake/3.16.4
\end{lstlisting}

\subsection{Additional Notes on GPU Accelerators}

As of January 2021, \texttt{CUDA} is the only GPU accelerator platform \exago fully supports.
We have preliminary support for \texttt{HIP}, however this is not in any of our main development branches yet.
Full \texttt{HIP} support should arrive in early 2021.

\subsection{Additional Notes on Umpire}

\texttt{Umpire} is an implicit dependency of \texttt{RAJA}.
If a user enables \texttt{RAJA}, they must also provide a valid installation of \texttt{Umpire}.
Additionally, if a user would like to run \exago with \texttt{RAJA} and without \texttt{CUDA}, they must provide a CPU-only build of \texttt{Umpire} since an \texttt{Umpire} build with \texttt{CUDA} enabled will link against \texttt{CUDA}.

\section{Building and installation}
\label{sec:building_and_installation}

\subsection{Default Build}

\exago may be built with a standard \cmake workflow:

\begin{lstlisting}[language=bash,caption={Example \cmake workflow}]
> cd $EXAGODIR
> export BUILDDIR=$PWD/build INSTALLDIR=$PWD/install
> mkdir $BUILDDIR $INSTALLDIR
> cd $BUILDDIR
> cmake .. -DCMAKE_INSTALL_PREFIX=$INSTALLDIR
> make install
\end{lstlisting}

Following sections will assume the user is following the basic workflow outlined above.

\textbf{Note:} For changes to the \cmake configuration to take effect, the code wil have to be rebuilt.

\subsection{Additional Options}

To enable additional options, \cmake variables may be defined via \cmake command line arguments, \texttt{ccmake}, or \texttt{cmake-gui}.
\cmake options specific to \exago have an \texttt{EXAGO\_} prefix.
For example, the following shell commands will build \exago with \texttt{MPI}:

\begin{lstlisting}[language=bash]
> cmake .. -DCMAKE_INSTALL_PREFIX=$INSTALLDIR -DEXAGO_ENABLE_MPI=ON
\end{lstlisting}

\exago's \cmake configuration will search the usual system locations for an \texttt{MPI} installation.

For dependencies not installed to a system-wide location, users may also directly specify the location of a dependency.
For example, this will build \exago with \texttt{IPOPT} enabled and installed to a user directory:

\begin{lstlisting}[language=bash]
> cmake .. \
    -DCMAKE_INSTALL_PREFIX=$INSTALLDIR \
    -DEXAGO_ENABLE_IPOPT=ON \
    -DIPOPT_DIR=/path/to/ipopt
\end{lstlisting}

Notice that the \cmake variable \texttt{IPOPT\_DIR} does not have an \texttt{EXAGO\_} prefix.
This is because the variables specifying locations often belong to external \cmake modules.
\cmake variables indicating installation directories do not have an \texttt{EXAGO\_} prefix.

Some \cmake options effect others.
This is especially common when the user enables \exago's GPU options.
For example, if the user enables \texttt{EXAGO\_ENABLE\_GPU} and \texttt{EXAGO\_ENABLE\_RAJA}, the user must provide a GPU-enabled \texttt{RAJA} installation.
\texttt{Umpire} is also an implicit dependency of \texttt{RAJA}, so if the user enables \texttt{EXAGO\_ENABLE\_GPU} they must \textbf{also} provide a GPU-enabled \texttt{Umpire} installation.

Below is a complete shell session on PNNL's cluster Newell in which a more complicated \exago configuration is built, where each dependency installation is explicitly passed to \cmake.
Environment modules specific to Newell are provided to make the example thorough, even though they are not likely to work on another machine.

\begin{lstlisting}[language=bash,caption={\exago build with all options enabled}]
> module load gcc/7.4.0
> module load cmake/3.16.4
> module load openmpi/3.1.5
> module load magma/2.5.2_cuda10.2
> module load metis/5.1.0
> module load cuda/10.2
> git clone https://gitlab.pnnl.gov/exasgd/frameworks/exago.git exago
> export EXAGODIR=$PWD/exago
> cd $EXAGODIR
> export BUILDDIR=$PWD/build INSTALLDIR=$PWD/install
> mkdir $BUILDDIR $INSTALLDIR
> cd $BUILDDIR
> cmake .. \
  -DCMAKE_INSTALL_PREFIX=$INSTALLDIR \
  -DCMAKE_BUILD_TYPE=Debug \
  -DEXAGO_ENABLE_GPU=ON \
  -DEXAGO_ENABLE_HIOP=ON \
  -DEXAGO_ENABLE_IPOPT=ON \
  -DEXAGO_ENABLE_MPI=ON \
  -DEXAGO_ENABLE_PETSC=ON \
  -DEXAGO_RUN_TESTS=ON \
  -DEXAGO_ENABLE_RAJA=ON \
  -DEXAGO_ENABLE_IPOPT=ON \
  -DIPOPT_DIR=/path/to/ipopt \
  -DRAJA_DIR=/path/to/raja \
  -Dumpire_DIR=/path/to/umpire \
  -DHIOP_DIR=/path/to/hiop \
  -DMAGMA_DIR=/path/to/magma \
  -DPETSC_DIR=/path/to/petsc
> make -j 8 install
> # For the following commands, a job scheduler command may be needed.
> # Run test suite
> make test
> # Run an ExaGO application:
> $INSTALLDIR/bin/opflow
\end{lstlisting}

\section{Applications}
\exago consists of applications for optimization. The different applications available with \exago are listed in Table \ref{tab:exago_apps}

\begin{table}[h]
  \caption{\exago applications}
  \begin{tabular}{|l|p{0.6\textwidth}|l|}
    \hline
    \textbf{Name} & \textbf{Description} & \textbf{Solvers} \\
    \hline
    \opflow & AC optimal power flow & \ipopt,\tao,\hiop\\ \hline
    \scopflow & Multi-period security-constrained AC optimal power flow & \ipopt \\ \hline
    \tcopflow & Multi-period AC optimal power flow & \ipopt \\ \hline
    \sopflow & Stochastic security-constrained multi-period AC optimal power flow & \ipopt \\
    \hline
    \pflow & AC power flow & \petsc \\ \hline
  \end{tabular}
  \label{tab:exago_apps}
\end{table}

Each application has the following format for execution
\begin{lstlisting}[language=bash]
  ./app <app_options>
\end{lstlisting}
Here, \lstinline{app_options} are the command line options for the application. Each application has many options through which the input files and the control options can be set for the application. All application options have the form \lstinline{-app_option_name} followed by the \lstinline{app_option_value}. 
For instance,
\begin{lstlisting}[language=bash]
  ./opflow -netfile case9mod.m -opflow_model POWER_BALANCE_POLAR -opflow_solver IPOPT
\end{lstlisting}
will execute \opflow application using \lstinline{case9mod.m} input file with the model \lstinline{POWER_BALANCE_POLAR} and \ipopt solver.

